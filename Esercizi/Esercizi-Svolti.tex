\documentclass{article}

\usepackage{cancel}
\usepackage{amsmath,amssymb}
\usepackage[includehead,nomarginpar]{geometry}
\usepackage{graphicx}
\usepackage{amsfonts} 
\usepackage{verbatim}
\usepackage{mathrsfs}  
\usepackage{lmodern}
\usepackage{braket}
\usepackage{bookmark}
\usepackage{fancyhdr}
\usepackage{romanbarpagenumber}
%\usepackage{minted}
%\usepackage{subfig}
\usepackage[italian]{babel}
\usepackage{float}
%\usepackage{wrapfig}
%\usepackage[export]{adjustbox}
\usepackage{contour}
\usepackage[normalem]{ulem}
\allowdisplaybreaks

\setlength{\headheight}{12.0pt}
\addtolength{\topmargin}{-12.0pt}
\graphicspath{ {../Immagini/} }

%% TODO add metadata
\hypersetup{
    colorlinks=true,
    linkcolor=black,
    pdftitle={Esercizi Svolti Analisi II},
    pdfauthor={Giacomo Sturm},
    pdfsubject={Analisi II},
    pdfkeywords={}
}

\newsavebox{\tempbox} %{\raisebox{\dimexpr.5\ht\tempbox-.5\height\relax}}
\makeatother
%% Derivate:
\newcommand{\df}{\mathrm{d}}
\newcommand{\diff}[2]{\displaystyle\frac{\df{#1}}{\df{#2}}}
\newcommand{\diffn}[3]{\displaystyle\frac{\df^{#3}{#1}}{\df{#2}^{#3}}}
\newcommand{\pdiff}[2]{\displaystyle\frac{\partial{#1}{\partial{#2}}}}
\newcommand{\pdiffn}[3]{\displaystyle\frac{\partial^{#3}{#1}{\partial{#2}^{#3}}}}

%% Integrali:
\newcommand{\intab}[4]{\displaystyle\int_{#1}^{#2}{#3}\df{#4}}
\newcommand{\intinf}[2]{\intab{-\infty}{+\infty}{#1}{#2}}
\newcommand{\intpinf}[2]{\intab{0}{+\infty}{#1}{#2}}
\newcommand{\intninf}[2]{\intab{-\infty}{0}{#1}{#2}}

%% Sommatorie:
\newcommand{\sumab}[4]{\displaystyle\sum_{#4=#1}^{#2}{#4}}
\newcommand{\suminf}[2]{\sumab{-\infty}{+\infty}{1}{2}}
\newcommand{\sumpinf}[2]{\sumab{0}{+\infty}{1}{2}}
\newcommand{\sumninf}[2]{\sumab{-\infty}{0}{1}{2}}

\renewcommand{\contentsname}{Indice}

\numberwithin{equation}{subsection}
\newcommand{\tageq}{\tag{\stepcounter{equation}\theequation}}
\AtBeginDocument{%
    \renewcommand{\figurename}{Fig.}
}
\renewcommand{\ULdepth}{1.8pt}
\contourlength{0.6pt}
\newcommand{\myuline}[1]{%
    \uline{\phantom{#1}}%
    \llap{\contour{white}{#1}}%
}
\fancypagestyle{link}{\fancyhf{}\renewcommand{\headrulewidth}{0pt}\fancyfoot[C]{Sorgente del file \LaTeX ed ultima versione del testo disponibile al link: \url{https://github.com/00Darxk/Analisi-II/}}}

\begin{document}

\title{%
    \textbf{Analisi II}  \\ 
    \large Esercizi Svolti di Analisi II \\
    \textit{Anno Accademico: 2024/25}}
\author{\textit{Giacomo Sturm}}
\date{\textit{Dipartimento di Ingegneria Industriale, Elettronica e Meccanica \\
Università degli Studi ``Roma Tre"}} 

\maketitle
\thispagestyle{link}

\clearpage


\pagestyle{fancy}
\fancyhead{}\fancyfoot{}
\fancyhead[C]{\textit{Analisi II - Università degli Studi ``Roma Tre"}}
\fancyfoot[C]{\thepage}
\pagenumbering{Roman}

\tableofcontents

\clearpage
\pagenumbering{arabic}

%% TODO inserire date per esercizi

\section{Serie Numeriche}

Studiare il carattere della serie numerica:
\begin{gather*}
    \displaystyle\sum_{k=1}^\infty\frac{k^2+1}{k^3+1}\ln\left(1+\frac{1}{k}\right)
\end{gather*}
Per il limite notevole, il carattere di questa serie è asintoticamente equivalente alla serie:
\begin{gather*}
    \displaystyle\sum_{k=1}^\infty\frac{k^2+1}{k^3+1}\ln\left(1+\frac{1}{k}\right)\sim
    \displaystyle\sum_{k=1}^\infty\frac{k^2+1}{k^3+1}\frac{1}{k}=
    \displaystyle\sum_{k=1}^\infty\frac{k^2+1}{k^4+k}
\end{gather*}
Il numeratore si comporta come $k^2$, sono asintoticamente equivalenti, mentre il denominatore è asintoticamente equivalente a $k^4$, quindi la serie è equivalente a:
\begin{gather*}
    \displaystyle\sum_{k=1}^\infty\frac{k^2+1}{k^4+k}\sim
    \displaystyle\sum_{k=1}^\infty\frac{\cancel{k^2}}{k^{\cancelto{2}{4}}}
\end{gather*}

Questa è una serie armonica generalizzata con $\alpha=2>1$, per cui converge. 


Si può realizzare una serie di stesso carattere sostituendo il fattore trascendente con un altro sempre appartenente ad un limite notevole:
\begin{gather*}
    \displaystyle\sum_{k=1}^\infty\frac{k^2+1}{k^3+1}\left(e^{1/k}-1\right)\sim
    \displaystyle\sum_{k=1}^\infty\frac{k^2+1}{k^3+1}\frac{1}{k}=
    \displaystyle\sum_{k=1}^\infty\frac{k^2+1}{k^4+k}\sim
    \displaystyle\sum_{k=1}^\infty\frac{\cancel{k^2}}{k^{\cancelto{2}{4}}}
\end{gather*}



Studiare il carattere della seguente serie:
\begin{gather*}
    \displaystyle\sum_{k=1}^\infty\frac{k^4+k^3}{k^5+1}\ln\left(\frac{k+1}{k}\right)=
    \displaystyle\sum_{k=1}^\infty\frac{k^4+k^3}{k^5+1}\ln\left(1+\frac{1}{k}\right)\sim
    \displaystyle\sum_{k=1}^\infty\frac{k^4+k^3}{k^5+1}\frac{1}{k}=
    \displaystyle\sum_{k=1}^\infty\frac{k^5+k^4}{k^6+k}\sim
    \displaystyle\sum_{k=1}^\infty\frac{\cancel{k^5}}{k^{\cancel{6}}}=
    \displaystyle\sum_{k=1}^\infty\frac{1}{k}
\end{gather*}
Essendo asintoticamente equivalente ad una serie armonica divergente, la serie diverge. 


Studiare il carattere della seguente serie:
\begin{gather*}
    \displaystyle\sum_{k=1}^\infty\frac{5^k}{k^2(2^k+4^k)}
\end{gather*}
Poiché il calcolo della condizione necessaria di convergenza è complessa, per risolvere l'esercizio si suppone sia corretta e si utilizza subito uno dei criteri:
\begin{gather*}
    \displaystyle\sum_{k=1}^\infty\frac{5^k}{k^2(2^k+4^k)}=
    \displaystyle\sum_{k=1}^\infty\frac{5^k}{k^24^k(1/2^k+1)}
\end{gather*}
Si mette in evidenza l'esponenziale con base maggiore, poiché il fattore con base minore di uno per $k\to\infty$ tende a zero, quindi questa serie è asintoticamente equivalente a:
\begin{gather*}
    \displaystyle\sum_{k=1}^\infty\frac{5^k}{k^24^k(1/2^k+1)}\sim
    \displaystyle\sum_{k=1}^\infty\frac{1}{k^2}\left(\frac{5}{4}\right)^k
\end{gather*}
Si utilizza ora il criterio del rapporto, si determina la serie ausiliaria:
\begin{gather*}
    \displaystyle\frac{\frac{1}{(k+1)^2}\left(\frac{5}{4}\right)^{\cancel{k+1}}}{\frac{1}{k^2}\cancel{\left(\frac{5}{4}\right)^k}}=\frac{5}{4}\left(\frac{k}{k+1}\right)^2\\
    \frac{5}{4}\lim_{k\to\infty}\left(\frac{k}{k+1}\right)^2=\frac{5}{4}>1
\end{gather*}
Quindi la serie diverge.  


Studiare il carattere della seguente serie:
\begin{gather*}
    \displaystyle\sum_{k=1}^\infty\frac{k^23^k}{2^k+4^k}
\end{gather*}
Questo esercizio è molto simile al precedente, ed analogamente si manipola il denominatore per rimuovere la somma di esponenziali:
\begin{gather*}
    \displaystyle\sum_{k=1}^\infty\frac{k^23^k}{2^k+4^k}=
    \displaystyle\sum_{k=1}^\infty\frac{k^23^k}{4^k(1/2^k+1)}\sim
    \displaystyle\sum_{k=1}^\infty\frac{k^23^k}{4^k}=
    \displaystyle\sum_{k=1}^\infty k^2\left(\frac{3}{4}\right)^k
\end{gather*}
Si utilizza analogamente alla precedente il criterio del rapporto; la sua serie ausiliaria è:
\begin{gather*}
    \displaystyle\frac{(k+1)^2\left(\frac{3}{4}\right)^{\cancel{k+1}}}{k^2\cancel{\left(\frac{3}{4}\right)^k}}=\frac{3}{4}\left(\frac{k+1}{k}\right)^2\\
    \lim_{k\to\infty}\frac{3}{4}\left(\frac{k+1}{k}\right)^2=\frac{3}{4}<1
\end{gather*}
Poiché il rapporto tendente all'infinito è minore di uno, allora la serie converge. 


Studiare il carattere della seguente serie:
\begin{gather*}
    \displaystyle\sum_{k=1}^\infty\frac{k^2+1}{k^3+1}\ln\left(\frac{k+1}{k}\right)
\end{gather*}
Questo è esattamente uguale al primo esercizio, quindi anch'esso diverge, l'argomento del logaritmo è scritto in una forma equivalente. 


Studiare il carattere della seguente serie:
\begin{gather*}
    \displaystyle\sum_{k=1}^\infty\frac{2^k(k^2+\sin e^k)}{3^k}\sim
    \displaystyle\sum_{k=1}^\infty\frac{2^kk^2}{3^k}=
    \displaystyle\sum_{k=1}^\infty k^2\left(\frac{2}{3}\right)
\end{gather*}
Nel numeratore si ha $k^2+\sin e^k$~$k^2$, poiché l'oscillazione di $\sin e^k$ viene smorzata all'infinito da $k^2$, può essere dimostrato tramite il criterio del confronto. 
Si applica il criterio del rapporto:
\begin{gather*}
    \displaystyle\frac{(k+1)^2\left(\frac{2}{3}\right)^{\cancel{k+1}}}{k^2\cancel{\left(\frac{2}{3}\right)^k}}=\frac{3}{4}\left(\frac{k+1}{k}\right)^2\\
    \lim_{k\to\infty}\frac{2}{3}\left(\frac{k+1}{k}\right)^2=\frac{2}{3}<1
\end{gather*}
Essendo il limite minore di uno, la serie converge. 

%% TODO SEGNARE EQUAZIONE

La formula di Stirling afferma che un fattoriale di un numero intero si comporta al limite per $k\to\infty$ come:
\begin{gather*}
    \displaystyle\lim_{k\to\infty}\frac{k!}{\sqrt{2k\pi}\left(\frac{k}{e}\right)^k}=1\\
    k!\sim\sqrt{2k\pi}\left(\frac{k}{e}\right)^k\tageq
\end{gather*}



Studiare il carattere della seguente serie:
\begin{gather*}
    \displaystyle\sum_{k=1}^\infty\frac{\sin k^3-k^{3/5}}{k^{1/4}\ln(k^k+k!)}
\end{gather*}
Il denominatore è sempre positivo, mentre nel numeratore il seno può essere al massimo pari ad uno, mentre l'altro fattore $k^{3/5}$ è sicuramente maggiore di uno da un certo $k_0$, quindi si può esprimere come:
\begin{gather*}
    -\displaystyle\sum_{k=1}^\infty\frac{k^{3/5}-\sin k^3}{k^{1/4}\ln(k^k+k!)}
\end{gather*}

Il numeratore può essere sostituito con $k^{3/5}$, analogamente alla precedente utilizzando il criterio del confronto può essere dimostrato come il seno diviso questo fattore tende a zero per $k\to\infty$, quindi la serie si può riscrivere come:
\begin{gather*}
    -\displaystyle\sum_{k=1}^\infty\frac{k^{3/5}-\sin k^3}{k^{1/4}\ln(k^k+k!)}\sim
    -\displaystyle\sum_{k=1}^\infty\frac{k^{3/5}}{k^{1/4}\ln(k^k+k!)}=
    -\displaystyle\sum_{k=1}^\infty \frac{k^{7/20}}{\ln(k^k+k!)}
\end{gather*}
Essendo $k^k$ un infinito di ordine superiore a $k!$, si può esprimere il denominatore in modo asintoticamente equivalente come:
\begin{gather*}
    -\displaystyle\sum_{k=1}^\infty \frac{k^{7/20}}{\ln(k^k+k!)}=
    -\displaystyle\sum_{k=1}^\infty \frac{k^{7/20}}{\ln\left[k^k\left(1+\frac{k!}{k^k}\right)\right]}=
    -\displaystyle\sum_{k=1}^\infty \frac{k^{7/20}}{\ln k^k+\ln\left(1+\frac{k!}{k^k}\right)}\sim
    -\displaystyle\sum_{k=1}^\infty \frac{k^{7/20}}{\ln k^k}    
\end{gather*}
Si dimostra:
\begin{gather*}
    \lim_{k\to\infty}\frac{\ln k^k+\ln\left(1+\frac{k!}{k^k}\right)}{\ln k^k}=
    \lim_{k\to\infty}\left[\cancelto{1}{\frac{\ln k^k}{\ln k^k}}+\frac{\ln\left(1+\frac{k!}{k^k}\right)}{\ln k^k}\right]\\
    \lim_{k\to\infty}\ln\left(1+\frac{k!}{k^k}\right)=0\rightarrow
    \lim_{k\to\infty}\frac{\ln k^k+\ln\left(1+\frac{k!}{k^k}\right)}{\ln k^k}=1
\end{gather*}
Si può riscrivere come:
\begin{gather*}
    -\displaystyle\sum_{k=1}^\infty\frac{k^{7/20}}{k}\frac{1}{\ln k}=
    -\displaystyle\sum_{k=1}^\infty\frac{1}{k^{13/20}\ln k}
\end{gather*}

In questa situazione si applica il criterio di condensazione creando la serie ausiliaria:
\begin{gather*}
    -\displaystyle\sum_{k=1}^\infty2^k\frac{1}{(2^k)^{13/20}\ln 2^k}=
    -\displaystyle\sum_{k=1}^\infty\frac{2^k}{(2^k)^{13/20}}\frac{1}{k\ln 2}=
    -\displaystyle\sum_{k=1}^\infty\frac{2^{7k/20}}{k\ln 2}=-\frac{1}{\ln 2}\sum_{k=1}^\infty\frac{2^{7k/20}}{k}
\end{gather*}
Applicando il criterio del rapporto si ha:
\begin{gather*}
    \displaystyle\frac{2^{\cancel{7(k+1/20)}}}{k+1}\frac{k}{\cancel{2^{7k/20}}}=2^{7/20}\frac{k}{k+1}\\
    \lim_{k\to\infty}2^{7/20}\frac{k}{k+1}=2^{7/20}<1
\end{gather*}
Poiché il limite tende ad un valore minore di uno, la serie converge. 



Studiare il carattere della serie:
\begin{gather*}
    \displaystyle\sum_{k=0}^\infty\left(\sqrt[3]{k^3+1}-k\right)
\end{gather*}

Si considera il prodotto notevole della differenza di due cubi, per determinare se questa serie soddisfa almeno la condizione necessaria per la convergenza:
\begin{gather*}
    \displaystyle a=\sqrt[3]{k^3+1}\land b=k\\
    \displaystyle a^3-a^3=(a-b)(a^2+ab+B^b)\\
    \displaystyle a-b=\frac{a^3-b^3}{a^2+ab+b^2}\\
    \displaystyle\sqrt[3]{k^3+1}-k=\frac{\cancel{k^3}+1-\cancel{k^3}}{\left(\sqrt[3]{k^3+1}\right)^2+k\sqrt[3]{k^3+1}+k^2}\\
    \displaystyle\sum_{k=0}^\infty\left(\sqrt[3]{k^3+1}-k\right)=
    \sum_{k=0}^\infty\frac{1}{\left(\sqrt[3]{k^3+1}\right)^2+k\sqrt[3]{k^3+1}+k^2}
\end{gather*}
Ragruppando gli addendi di potenza più grande:
\begin{gather*}
    \sum_{k=0}^\infty\frac{1}{\left(\sqrt[3]{k^3+1}\right)^2+k\sqrt[3]{k^3+1}+k^2}
    \displaystyle\sum_{k=0}^\infty\frac{1}{\sqrt[3]{k^6\left(1+\frac{2}{k^3}+\frac{1}{k^6}\right)}+k\sqrt[3]{k^3\left(1+\frac{1}{k^3}\right)}+k^2}\\
    \sum_{k=0}^\infty\frac{1}{k^2\left(\sqrt[3]{1+\frac{2}{k^3}+\frac{1}{k^6}}+\sqrt[3]{1+\frac{1}{k^3}}+1\right)}
    \sim\sum_{k=0}^\infty\frac{1}{k^2}
\end{gather*}

Al denominatore tutti i termini sono asintoticamente equivalenti a $k^2$, per cui questa serie è asintoticamente equivalente ad una serie armonica generalizzata con $\alpha=2$, a meno di una costante:
\begin{gather*}
    \displaystyle\sum_{k=0}^\infty\frac{1}{\left(\sqrt[3]{k^3+1}\right)^2+k\sqrt[3]{k^3+1}+k^2}
    \sim\sum_{k=0}^\infty\frac{1}{k^2}
\end{gather*} 
Quindi converge. 


Studiare la seguente serie:
\begin{gather*}
    \displaystyle\sum_{k=1}^\infty\frac{3^kk!}{k^k}
\end{gather*}
Considerando la formula di Stirling, si ottiene la serie asintoticamente equivalente:
\begin{gather*}
    \displaystyle\sum_{k=0}^\infty\frac{3^k}{k^k}\sqrt{2\pi k}\left(\frac{k}{e}\right)^k=\sum_{k=0}^\infty\left(\frac{3}{e}\right)^k\sqrt{2\pi k}
\end{gather*}
Utilizzando il criterio del rapporto, si costruisce la successione ausiliaria: 
\begin{gather*}
    \displaystyle\left(\frac{3}{e}\right)^{\cancel{k+1}}\cancel{\sqrt{2\pi}} \sqrt{(k+1)}\cancel{\left(\frac{e}{3}\right)^k}\frac{1}{\cancel{\sqrt{2\pi}} \sqrt{k}}=
    \frac{3}{e}\frac{\sqrt{k+1}}{\sqrt{k}}\\
    \lim_{k\to\infty}\frac{3}{e}\sqrt{\frac{k+1}{k}}=\frac{3}{e}>1
\end{gather*}
Quindi questa serie converge. 


Studiare il carattere della seguente serie, per $\alpha\in\mathbb{R}^+$:
\begin{gather*}
    \displaystyle\sum_{k=1}^\infty k^{2/3-3\alpha}\left(e^{1/k^{2\alpha}}-1\right)
\end{gather*}
Si considera il limite notevole $e^{1/k^{2\alpha}}-1\sim1/k^{2\alpha}$ e si considera la serie asintoticamente equivalente:
\begin{gather*}
    \displaystyle\sum_{k=1}^\infty k^{2/3-3\alpha}\left(e^{1/k^{2\alpha}}-1\right)\sim
    \displaystyle\sum_{k=1}^\infty \frac{k^{2/3-3\alpha}}{k^{2\alpha}}=\sum_{k=1}^\infty k^{2/3-5\alpha}
\end{gather*}
Si può rendere equivalente a duna serie armonica generalizzata:
\begin{gather*}
    \displaystyle\sum_{k=1}^\infty\frac{1}{k^{5\alpha-2/3}}:\begin{cases}
        5\alpha-\displaystyle\frac{\strut 2}{\strut 3}>1 &\mbox{converge}\\
        5\alpha-\displaystyle\frac{\strut 2}{\strut 3}<1 &\mbox{diverge}
    \end{cases}=\begin{cases}
        \alpha>\displaystyle\frac{\strut 1}{\strut 3} &\mbox{converge}\\
        \alpha<\displaystyle\frac{\strut 1}{\strut 3} &\mbox{diverge}
    \end{cases}
\end{gather*}


Studiare il carattere della seguente serie, per $\alpha\in\mathbb{R}^+$:
\begin{gather*}
    \displaystyle\sum_{k=1}^\infty k^{1/2-\alpha}\ln\left(1+\frac{1}{k^{3\alpha}}\right)\sim\sum_{k=1}^\infty k^{1/2-\alpha}\frac{1}{k^{3\alpha}}=\sum_{k=1}^\infty\frac{1}{k^{4\alpha-1/2}}
\end{gather*}

La serie converge per $4\alpha-1/2>1$




Studiare il carattere della serie:
\begin{gather*}
    \displaystyle\sum_{k=1}^\infty\frac{(-1)^k}{k}
\end{gather*}
La serie rispetta il criterio di Leibniz e quindi converge. 


Studiare il carattere della serie:
\begin{gather*}
    \displaystyle\sum_{k=1}^{\infty}(-1)^k\frac{2^{k+1}}{3^{k+2}k!}=
    \sum_{k=1}^\infty(-1)^k\left(\frac{2}{3}\right)^k\frac{2}{9}\frac{1}{k!}=\frac{2}{9}\sum_{k=1}^\infty(-1)^k\left(\frac{2}{3}\right)^k\frac{1}{k!}
\end{gather*}



Studiare il carattere della seguente serie, per $\alpha\in\mathbb{R}$:
\begin{gather*}
    \displaystyle\sum_{k=1}^\infty\frac{k^{2\alpha}\cos(k\pi)}{k+1}
\end{gather*}
Avendo $\pi$ all'interno del coseno, ed essendo $k$ solo numeri naturali, il fattore coseno alterna segno, mentre di modulo vale sempre uno:
\begin{gather*}
    \displaystyle\sum_{k=1}^\infty\frac{k^{2\alpha}\cos(k\pi)}{k+1}=
    \sum_{k=1}^\infty(-1)^k\frac{k^{2\alpha}}{k+1}
\end{gather*}

Si considera la funzione associata $f(x)$, e si studia la sua derivata per determinare la sua monotonia:
\begin{gather*}
    f(x)=\displaystyle\frac{x^{2\alpha}}{x+1}\\
    \displaystyle\frac{\mathrm{d}f(x)}{\mathrm{d}x}=\frac{2\alpha x^{2\alpha-1}(x+1)-x^{2\alpha}}{(x+1)^2}=
    \frac{(2\alpha-1)x^{2\alpha}+2\alpha x^{2\alpha-1}}{(x+1)^2}=\frac{x^{2\alpha-1}}{x+1}\left[(2\alpha-1)x+2\alpha\right]\\\frac{x^{2\alpha-1}}{x+1}\left[(2\alpha-1)x+2\alpha\right]<0:\mbox{converge}:\,2\alpha-1<0\implies\alpha<\frac{1}{2}
\end{gather*}


Studiare il carattere della seguente serie:
\begin{gather*}
    \displaystyle\sum_{k=1}^\infty(-1)^k\frac{\sqrt{k}}{k+1}
\end{gather*}
Si vuole trovare un'approssimazione di $s$ a meno di $10^{-3}$:
\begin{gather*}
    |s-s_n|\leq|a_n|=\displaystyle\frac{\sqrt{n}}{n+1}<10^{-3}\\
    \text{per }n=10^6\Rightarrow\frac{10^6}{10^6-1}=0.000\bar{9}<10^{-3}
\end{gather*}

Le prime tre cifre della somma per $n=10^6$ coincidono al valore reale della somma. 




Studiare il carattere della seguente serie:
\begin{gather*}
    \displaystyle\sum_{k=1}^\infty\frac{\sin k}{k^2}
\end{gather*}
Per il criterio di convergenza assoluta:
\begin{gather*}
    \displaystyle\sum_{k=1}^\infty\left|\frac{\sin k}{k^2}\right|
\end{gather*}
Per il criterio del confronto si ha:
\begin{gather*}
    \displaystyle\frac{\sin k}{k^2}\leq\frac{1}{k^2}
\end{gather*}
La serie maggiorante è una serie armonica generalizzata convergente, quindi la serie iniziale converge. 




Studiare l'insieme di convergenza di questa serie:
\begin{gather*}
    \displaystyle\sum_{k=0}^\infty x^k
\end{gather*}
Ha centro in $x_0$, e coefficienti unitari $a_k=1$. Si determina il valore del raggio di convergenza:
\begin{gather*}
    R=\lim_{k\to\infty}\left|\frac{a_k}{{a_{k+1}}}\right|=1
\end{gather*}
La serie converge sicuramente nell'intervallo di convergenza: $(-1,+1)$. Ci sono due casi dubbi nei punti $x=\pm1$:
\begin{gather*}
    \displaystyle\sum_{k=0}^\infty 1^k=\sum_{k=1}^\infty1=+\infty
\end{gather*}
La serie non converge per $x=1$, mentre in $x=-1$:
\begin{gather*}
    \displaystyle\sum_{k=0}^\infty (-1)^k
\end{gather*}
Questa serie a termini di segno alterno non soddisfa la condizione necessaria di convergenza, quindi non converge neanche in $x=-1$. 
Quindi in questo caso l'insieme di convergenza coincide all'intervallo di convergenza. 

È possibile esprimere la funzione $1/(1-x)$ in termini di questa serie, se $x$ appartiene all'insieme di convergenza. Si useranno serie, come la serie di Taylor per approssimare il comportamento di funzioni trascendenti in forma di polinomio. 



Studiare l'insieme di convergenza della seguente serie di potenze:
\begin{gather*}
    \displaystyle\sum_{k=1}^\infty\frac{x^k}{k}
\end{gather*}
È sempre centrata in $x_0=0$, ma i coefficienti non sono costanti $a_k=k^{-1}$. 
Si determina il valore del raggio di convergenza:
\begin{gather*}
    R=\lim_{k\to\infty}\left|\frac{a_k}{{a_{k+1}}}\right|=
    \lim_{k\to\infty}\left|\frac{k^{-1}}{{(k+1)^{-1}}}\right|=
    \lim_{k\to\infty}\frac{k+1}{k}=1
\end{gather*}
Quindi sicuramente questa serie converge nell'intervallo $(-1,+1)$. Bisogna studiare il carattere nei valori di frontiera:
\begin{gather*}
    x=1\rightarrow\displaystyle\sum_{k=1}^\infty\frac{1^k}{k}=\sum_{k=1}^\infty\frac{1}{k}=+\infty\\
    x=-1\rightarrow\displaystyle\sum_{k=1}^\infty\frac{(-1)^k}{k}:\text{converge}
\end{gather*}
La serie converge perché è una serie di segno alterno, con $1/k$ decrescente tendente a zero. Quindi la serie converge in $[-1,1)$. 



Studiare l'insieme di convergenza della seguente serie di potenze:
\begin{gather*}
    \displaystyle\sum_{k=1}^\infty\frac{x^k}{k^2}
\end{gather*}
È sempre centrata in $x_0=0$, dove il termine generico è $a_k=k^{-2}$. 
Si determina il valore del raggio di convergenza:
\begin{gather*}
    R=\lim_{k\to\infty}\left|\frac{a_k}{{a_{k+1}}}\right|=
    \lim_{k\to\infty}\left|\frac{k^{-2}}{{(k+1)^{-2}}}\right|=
    \lim_{k\to\infty}\frac{(k+1)^2}{k^2}=1
\end{gather*}
Quindi sicuramente questa serie converge nell'intervallo $(-1,+1)$. Bisogna studiare il carattere nei valori di frontiera:
\begin{gather*}
    x=1\rightarrow\displaystyle\sum_{k=1}^\infty\frac{1^k}{k^2}=\sum_{k=1}^\infty\frac{1}{k^2}:\text{converge}\\
    x=-1\rightarrow\displaystyle\sum_{k=1}^\infty\frac{(-1)^k}{k^2}:\text{converge}
\end{gather*}
La prima converge poiché è una serie armonica generalizzata con $\alpha>1$, mentre la seconda serie converge perché è una serie di segno alterno, con $1/k$ decrescente tendente a zero, per il criterio di Leibniz. Quindi la serie converge in $[-1,1]$. 




Studiare l'insieme di convergenza della seguente serie:
\begin{gather*}
    \displaystyle\sum_{k=1}^\infty\frac{k^2}{3^k}x^k
\end{gather*}
Si determina il raggio di convergenza della serie:
\begin{gather*}
    R=\lim_{k\to\infty}\left|\frac{a_k}{{a_{k+1}}}\right|=
    \lim_{k\to\infty}\frac{\frac{k^2}{3^k}}{\frac{(k+1)^2}{3^{k+1}}}=
    \lim_{k\to\infty}\frac{k^23^{\cancel{k+1}}}{\cancel{3^k}(k+1)^2}=3
\end{gather*}
Sicuramente la serie di potenze converge nell'intervallo di convergenza $(-3, 3)$. Si determina il comportamento agli estremi:
\begin{gather*}
    x=3\rightarrow\displaystyle\sum_{k=1}^\infty\frac{k^2}{\cancel{3^k}}\cancel{3^k}=+\infty\\
    x=-3\rightarrow\displaystyle\sum_{k=1}^\infty\frac{k^2}{3^k}(-3)^k=
    \sum_{k=1}^\infty(-1)^k\frac{k^2}{\cancel{3^k}}\cancel{3^k}=+\infty
\end{gather*}
In questo caso l'insieme di convergenza coincide con l'intervallo di convergenza: $(-3, 3)$. 



Studiare il carattere della seguente serie:
\begin{gather*}
    \displaystyle\sum_{k=1}^\infty\frac{1}{k}\left(\frac{x}{4}\right)^k
\end{gather*}
Con un cambiamento di base si può riportare questa serie in una serie di potenze con centro nell'origine: $t=x/4$:
\begin{gather*}
    \displaystyle\sum_{k=1}^\infty\frac{t^k}{k}
\end{gather*}
Questa serie è già stata analizzata precedentemente, e converge con $t\in[-1,1)$. Ma bisogna sempre fare riferimento alla variabile originale, quindi la serie originale converge per $x\in[-4, 4)$. 


Studiare il carattere della seguente serie:
\begin{gather*}
    \displaystyle\sum_{k=1}^\infty\frac{k!2^k+5}{(k+3)!}(x+2)^k
\end{gather*}
Questa serie di potenze è centrata in $x_0=-2$, con termini generici:
\begin{gather*}
    \displaystyle\frac{k!2^k+5}{(k+3)!}
\end{gather*}
Si preferisce lavorare con serie di potenze con centro nell'origine, per cui si effettua il cambio di variabile $t=x+2$, mentre il termine generico rimane invariato:
\begin{gather*}
    \displaystyle\sum_{k=1}^\infty\frac{k!2^k+5}{(k+3)!}t^k
\end{gather*}
% Si considera la serie asintoticamente equivalente:
% \begin{gather*}
%     \displaystyle\sum_{k=1}^\infty\frac{k!2^k+5}{(k+3)!}t^k\sim
%     \displaystyle\sum_{k=1}^\infty\frac{k!}{(k+3)!}2^kt^k
% \end{gather*}

Si determina il valore del raggio di convergenza:
\begin{gather*}
    R=\displaystyle\lim_{k\to\infty}\frac{a_k}{a_{k+1}}=\frac{\frac{k!2^k+5}{(k+3)!}}{\frac{(k+1)!2^{k+1}+5}{(k+4)!}}=
    \lim_{k\to\infty}\frac{(k!2^k+5)\cancelto{k+4}{(k+4)!}}{[(k+1)!2^{k+1}+5]\cancel{(k+3)!}}=
    \lim_{k\to\infty}\frac{k!2^k+5}{(k+1)!2^{k+1}+5}(k+4)
\end{gather*}
Dato che si è scomposto in fattori si possono considerare fattori asintoticamente equivalenti:
\begin{gather*}
    k!2^k+5\sim k!2^k\\
    (k+1)!2^{k+1}+5\sim(k+1)!2^{k+1}
\end{gather*}
Il limite diventa:
\begin{gather*}
    \lim_{k\to\infty}\frac{k!2^k+5}{(k+1)!2^{k+1}+5}(k+4)=
    \lim_{k\to\infty}\frac{\cancel{k!}2^k}{\cancelto{k+1}{(k+1)!}2^{k+1}}(k+4)=
    \lim_{k\to\infty}\frac{k+4}{k+1}\frac{\cancel{2^k}}{2^{\cancel{k+1}}}=\frac{1}{2}    
\end{gather*}
La serie converge nell'intervallo: $(-1/2, 1/2)$. Si analizza il suo comportamento agli estremi:
\begin{gather*}
    \displaystyle t=\frac{1}{2}\rightarrow\displaystyle\sum_{k=1}^\infty\frac{k!2^k+5}{(k+3)!}\frac{1}{2^k}\sim
    \displaystyle\sum_{k=1}^\infty\frac{k!\cancel{2^k}}{(k+3)!}\frac{1}{\cancel{2^k}}=
    \sum_{k=1}^\infty\frac{1}{(k+3)(k+2)(k+1)}\sim\sum_{k=1}^\infty\frac{1}{k^3}:\text{converge}\\
    \displaystyle t=-\frac{1}{2}\rightarrow\displaystyle\sum_{k=1}^\infty\frac{k!2^k+5}{(k+3)!}\frac{(-1)^k}{2^k}\sim\sum_{k=1}^\infty\frac{(-1)^k}{k^3}:\text{converge}
\end{gather*}
La prima è una serie armonica generalizzata con $\alpha>1$, quindi converge, la seconda è una serie di segni alterni che soddisfa il criterio di Leibniz, quindi converge. L'intervallo di convergenza è $[-1/2, 1/2]$. La serie originale quindi converge per $x\in[-5/2, 3/2]$. 


Determinare l'insieme di convergenza della seguente serie e determinarne la somma:
\begin{gather*}
    \displaystyle\sum_{k=0}^\infty\left(\frac{1}{1-\ln|x|}\right)^k
\end{gather*}
Per $x\in\mathbb{R}$, questa serie rappresenta una serie di potenze particolare, una serie geometrica con ragione $t$, convergente per $t\in(-1,1)$:
\begin{gather*}
    t=\displaystyle\frac{1}{1-\ln|x|}\\
    \begin{cases}
        \displaystyle\frac{\strut 1}{\strut 1-\ln|x|}< -1\\
        \displaystyle\frac{\strut 1}{\strut 1-\ln|x|}> 1
    \end{cases}
\end{gather*}
% Con somma $s$:
% \begin{gather*}
    
% \end{gather*}
%% TODO finire esercizio

%% SOLUZIONE
\begin{gather*}
    A=(-\infty, -e^2)\cup(-1,0)\cup(0,1)\cup(e^2,\infty)
\end{gather*}


Studiare l'insieme di convergenza della seguente serie:
\begin{gather*}
    \displaystyle\sum_{k=1}^\infty\frac{e^{1/k}-1}{3\ln k+1}(x-5)^k
\end{gather*}
Questa è una serie di potenze con centro in $x_0=5$, si effettua una sostituzione per centrarla nell'origine, $x-5=t$:
\begin{gather*}
    \displaystyle\sum_{k=1}^\infty\frac{e^{1/k}-1}{3\ln k+1}t^k
\end{gather*}
Questa è una serie con termini a segno costante positivo, poiché per ogni $k>1$, si ha che $e^{1/k}>1$. Si determina la successione ausiliaria:
\begin{gather*}
    \displaystyle\left|\frac{a_k}{a_{k+1}}\right|=
    \frac{\frac{e^{k^{-1}}-1}{3\ln k+1}}{\frac{e^{(k+1)^{-1}}-1}{3\ln(k+1)+1}}=\frac{e^{k^{-1}}-1}{\cancel{3}\ln k+1}\frac{\cancel{3}\ln(k+1)+1}{e^{(k+1)^{-1}}-1}
\end{gather*}
Entrambi sono di segno positivo, quindi si può togliere il modulo essendo di segno positivo, si ha che $e^{t}-1\sim t$, analogamente per il logaritmo si ha $\ln t+c\sim\ln t$:
\begin{gather*}
    \frac{e^{k^{-1}}-1}{\ln k}\frac{\ln(k+1)+1}{e^{(k+1)^{-1}}-1}\sim\frac{k^{-1}}{\ln k+1}\frac{\ln(k+1)}{(k+1)^{-1}}=\frac{k+1}{k}\frac{\ln(k+1)}{\ln k}
\end{gather*}
Si determina il raggio di convergenza:
\begin{gather*}
    R=\lim_{k\to\infty}\frac{k+1}{k}\frac{\ln(k+1)}{\ln k}=\cancelto{1}{\lim_{k\to\infty}\frac{k+1}{k}}\lim_{k\to\infty}\frac{\ln(k+1)}{\ln k}
\end{gather*}
Per il teorema di de l'Hopital si ha:
\begin{gather*}
    R=\lim_{k\to\infty}\frac{\ln(k+1)}{\ln k}=\lim_{k\to\infty}\frac{k}{k+1}=1
\end{gather*}
Il raggio di convergenza è quindi pari ad uno. L'intervallo di convergenza di questa serie è l'intervallo di estremi $(-1, 1)$, questo fa riferimento alla variabile $t$, in $x$ questo intervallo è $(4,6)$. Si determina ora il comportamento agli estremi:
\begin{gather*}
    t=1\implies
    \displaystyle\sum_{k=1}^\infty\frac{e^{1/k}-1}{3\ln k+1}\cancel{1^k}\sim
    \displaystyle\frac{1}{3}\sum_{k=1}^\infty\frac{1}{k\ln k}\sim\frac{1}{3}\sum_{k=1}^\infty\frac{\cancel{2^k}}{\cancel{2^k}\ln2^k}=\frac{1}{3\ln2}\sum_{k=1}^\infty\frac{1}{k}\to\infty
    \\
    t=-1\implies
    \displaystyle\sum_{k=1}^\infty\frac{e^{1/k}-1}{3\ln k+1}(-1)^k
\end{gather*}
Per il primo caso si usano le funzioni asintoticamente equivalenti individuate nel calcolo del raggio di convergenza, per il secondo caso si usa il criterio di Leibniz, ma bisogna dimostrare che i termini della successione siano definitivamente decrescenti a zero. Il numeratore è un'esponenziale e decresce a zero per $k\to\infty$, il numeratore tende a zero, mentre il logaritmo cresce tendente a $+\infty$. Quindi questa serie converge per $t=-1$, l'insieme di convergenza diventa $[-1,1)\to[4,6)$. 



Studiare il carattere della seguente serie:
\begin{gather*}
    \displaystyle\sum_{k=1}^\infty\left(\frac{k-1}{k+1}\right)^{k^2}x^{4k}
\end{gather*}
Questa serie si può trasformare in una serie di potenze con centro nell'origine, effettuando la sostituzione $t=x^4$:
\begin{gather*}
    \displaystyle\sum_{k=1}^\infty\left(\frac{k-1}{k+1}\right)^{k^2}t^{k}
\end{gather*}
Conviene invece di usare la successione ausiliaria simile a quella del criterio del rapporto, si utilizza la successione ausiliaria:
\begin{gather*}
    \displaystyle\frac{1}{\sqrt[k]{a_k}}=\frac{1}{\sqrt[\cancel{k}]{\left(\frac{k-1}{k+1}\right)^{k^2}}}
\end{gather*}
Si determina il raggio di convergenza:
\begin{gather*}
    R=\lim_{k\to\infty}
    \frac{1}{\sqrt[k]{\left(\frac{k-1}{k+1}\right)^{k^2}}}=
    \lim_{k\to\infty}\left(\frac{k+1}{k-1}\right)^k=
    \lim_{k\to\infty}\left(1+\frac{2}{k-1}\right)^k=
    \lim_{k\to\infty}\left(1+\frac{2}{k-1}\right)^{k+1-1}\\
    \lim_{k\to\infty}\left(1+\frac{2}{k-1}\right)^{k-1}\cancelto{1}{\left(1+\frac{2}{k-1}\right)}=e^2
\end{gather*}
L'intervallo di convergenza è quindi $(-e^2,e^2)$, si considerano gli estremi:
\begin{gather*}
    t=e^2\implies\displaystyle\sum_{k=1}^\infty\left(\frac{k-1}{k+1}\right)^{k^2}e^{2k}\\
    t=-e^2\implies\displaystyle\sum_{k=1}^\infty\left(\frac{k-1}{k+1}\right)^{k^2}(-e)^{2k}
\end{gather*}
Poiché utilizzare uno dei criteri è abbastanza complesso, si analizza la condizione necessaria di convergenza:
\begin{gather*}
    \lim_{k\to\infty}\left(\frac{k-1}{k+1}\right)^{k^2}e^{2k}=
    \lim_{k\to\infty}e^{\ln\left(\frac{k-1}{k+1}\right)^{k^2}}e^{2k}=
    \lim_{k\to\infty}e^{k^2\ln\left(1-\frac{2}{k+1}\right)+2k}=
    e^{\lim_{k\to\infty}k^2\ln\left(1-\frac{2}{k+1}\right)+2k}
\end{gather*}
Si utilizza la formula di Maclaurin, se il limite dell'esponente tende a zero, il limite tende ad uno e quindi la serie non può convergere, si controlla questo limite:

\begin{gather*}
    \displaystyle\lim_{k\to\infty}k^2\ln\left(1-\frac{2}{k+1}\right)+2k\sim
    \lim_{k\to\infty}k^2\left(-\frac{2}{k+1}-\frac{2}{(k+1)^2}+o\left[\frac{1}{k^2}\right]\right)+2k\\
    \lim_{k\to\infty}-\frac{2k^2}{k+1}-\frac{2k^2}{(k+1)^2}+\cancelto{0}{\frac{o\left[\frac{1}{k^2}\right]}{\frac{1}{k^2}}}+2k=
    \lim_{k\to\infty}\frac{-2k^2(k+1)-2k^2+2k(k+1)^2}{(k+1)^2}\\
    \lim_{k\to\infty}\frac{\cancel{-2k^3}\bcancel{-2k^2}\bcancel{-2k^2}+\cancel{2k^3}+\bcancel{4k^2}+2k}{(k+1)^2}=
    \lim_{k\to\infty}{2k}{(k+1)^2}=0
\end{gather*}

L'esponente è un infinitesimo, quindi l'esponente tende ad uno, e quindi la serie non può convergere. Per $t=-e^2$ invece si ha una serie di segno alterno:
\begin{gather*}
    \displaystyle\sum_{k=1}^\infty(-1)^k\left(\frac{k-1}{k+1}\right)^{k^2}e^{2k}
\end{gather*}
Per converge la condizione necessaria deve valere, ma i termini di questa serie tendono ad uno, quindi questa serie si avvicina oscillando ad uno, ma non può convergere. Quindi l'intervallo di convergenza coincide con l'insieme di convergenza, quindi si ha che l'intervallo di convergenza in $x$ è dato da:
\begin{gather*}
    -e^2<x^4<e^2\implies x^4<e^2:-e<x^2<e\implies x^2<e:-\sqrt{e}<x<\sqrt{e}
\end{gather*}


Studiare la convergenza e la somma della seguente serie:
\begin{gather*}
    \displaystyle\sum_{k=1}^\infty\frac{3^k}{k4^{k+1}}
\end{gather*}
Si può riscrivere come:
\begin{gather*}
    \displaystyle\frac{1}{4}\sum_{k=1}^\infty\frac{1}{k}\left(\frac{3}{4}\right)^k
\end{gather*}
Questa è molto simile alla serie con $x\in\mathbb{R}$, a meno di una costante:
\begin{gather*}
    \displaystyle\frac{1}{4}\sum_{k=1}^\infty\frac{x^k}{k}=-\frac{1}{4}\ln(1-x)
\end{gather*}
Questa identità vale solamente se $x\in[-1,1)$, ed è verificata per il valore $3/4$. La serie di partenza quindi converge e si può esprimere come:
\begin{gather*}
    \displaystyle\frac{1}{4}\sum_{k=1}^\infty\frac{1}{k}\left(\frac{3}{4}\right)^k=
    -\frac{1}{4}\ln\left(1-\frac{3}{4}\right)=-\frac{1}{4}\ln\left(\frac{1}{4}\right)=
    \frac{1}{4}\ln4=\frac{\ln2}{2}=\ln\sqrt{2}
\end{gather*}


\clearpage

\section{Integrali Impropri}

Studiare la convergenza del seguente integrale:
\begin{gather*}
\displaystyle\int_0^1\frac{1+x}{\sqrt{x}}\mathrm{d}x
\end{gather*}
La funzione presente un punto critico in $x=0$, la funzione integranda per $x\to0^+$ si può stabilire sia asintoticamente equivalente alla funzione:
\begin{gather*}
    \displaystyle\frac{1+x}{\sqrt{x}}\sim\frac{1}{\sqrt{x}}
\end{gather*}
La funzione integranda è asintoticamente equivalente alla funzione campione $x^{-1/2}$, ed è noto sia convergente, quindi l'integrale improprio della funzione di partenza converge. 


Stabilire la convergenza del seguente integrale:
\begin{gather*}
    \displaystyle\int_0^1\frac{\mathrm{d}x}{\sqrt{x(1-x)}}
\end{gather*}
La funzione integranda presente due punti di singolarità per $x=0$ e $x=1$, si applica la proprietà additiva dell'integrale:
\begin{gather*}
    \displaystyle\int_0^1\frac{\mathrm{d}x}{\sqrt{x(1-x)}}=
    \overbrace{\int_0^a\frac{\mathrm{d}x}{\sqrt{x(1-x)}}}^{I_1}+
    \underbrace{\int_a^1\frac{\mathrm{d}x}{\sqrt{x(1-x)}}}_{I_2}\,\mbox{dove }0<a<1
\end{gather*}
Per convergere questo integrale, devono convergere entrambi gli integrali. Si considera il primo:
\begin{gather*}
    I_1=\displaystyle\int_0^a\frac{\mathrm{d}x}{\sqrt{x(1-x)}}
\end{gather*}
La sua funzione integranda è asintoticamente equivalente per $x\to0^+$:
\begin{gather*}
    \displaystyle\frac{1}{\sqrt{x}\sqrt{1-x}}\sim\frac{1}{\sqrt{x}}
\end{gather*}
Ed è noto che l'integrale improprio della funzione campione converge, quindi $I_1$ converge, mentre per la seconda funzione integranda, per $x\to1^-$:
\begin{gather*}
    \displaystyle\frac{1}{\sqrt{x}\sqrt{1-x}}\sim\frac{1}{(1-x)^{1/2}}
\end{gather*}
L'integrale improprio della funzione campione $(1-x)^{-1/2}$ è noto convergere, per un'esponente minore di uno, quindi anche l'integrale $I_2$, quindi l'integrale di partenza $I_1+I_2$ converge. 


Determinare la convergenza del seguente integrale improprio:
\begin{gather*}
    \displaystyle\int_0^2\frac{1}{x^3\sin x}\mathrm{d}x
\end{gather*}
La funzione integranda presenta un punto di singolarità per $x=0$, si considera per $x\to0^+$ la funzione asintoticamente equivalente:
\begin{gather*}
    \displaystyle\frac{1}{x^3\sin x}\sim\frac{1}{x^4}
\end{gather*}
Per il limite notevole $\sin x/x$, l'integrale improprio di questa funzione non converge, quindi la funzione integranda non è integrabile nell'intervallo $[0,2]$. 
Analogamente si ha lo stesso risultato per la funzione integranda $1/\sin x$. Per renderla integrabile bisogna elevare l'argomento del seno ad una potenza minore di uno:
\begin{gather*}
    \displaystyle\int_0^2\frac{1}{\sin x^\alpha}\mathrm{d}x
\end{gather*}
Dove $0<\alpha<1$, allora l'integrale improprio di seconda specie converge. 



Considerare il seguente integrale improprio e valutarne la convergenza:
\begin{gather*}
    \displaystyle\int_1^2\frac{1}{\ln x}\mathrm{d}x
\end{gather*}
La funzione integranda presenta un punto critico per $x=1$, si considera $x\to1^+$:
\begin{gather*}
    \ln x\sim x-1
\end{gather*}
Si dimostra considerando il seguente limite:
\begin{gather*}
    \lim_{x\to1^+}\frac{\ln x}{x-1}
\end{gather*}
Utilizzando il teorema di de l'Hopital:
\begin{gather*}
    \lim_{x\to1^+}\frac{\frac{1}{x}}{1}=1
\end{gather*}
La funzione integranda per $x\to1^+$ è asintoticamente equivalente a:. 
\begin{gather*}
    \displaystyle\frac{1}{\ln x}\sim\frac{1}{x-1}
\end{gather*}
Questa funzione è divergente, poiché $\alpha=1$



Studiare la convergenza del seguente integrale improprio:
\begin{gather*}
    \displaystyle\int_1^{2}\frac{1}{x\ln x}\mathrm{d}x
\end{gather*}
Il punto di singolarità è $x=1$, per la funzione logaritmo che tende il denominatore a zero. Per $x\to1^+$ si ha:
\begin{gather*}
    \ln x\sim x-1\\
    x\sim 1
\end{gather*}
Quindi si ha un integrale analogo al precedente, ed anch'esso diverge. 
Si considera ora l'integrale nell'intervallo $[0,1/2]$:
\begin{gather*}
    \displaystyle\int_0^{1/2}\frac{1}{x\ln x}\mathrm{d}x
\end{gather*}
Il punto di singolarità è $x=0$, non si riesce a trovare una funzione asintoticamente equivalente, ma si prova a verificare in maniera diretta mediante la definizione: 
\begin{gather*}
    \displaystyle\int_0^{1/2}\frac{1}{x\ln x}\mathrm{d}x=
    \lim_{\varepsilon\to0^+}\int_{0+\varepsilon}^{1/2}\frac{\mathrm{d}(\ln x)}{\ln x}=\ln |\ln x|\bigg|_{0+\varepsilon}^{1/2}=\lim_{\varepsilon\to0^+}\left\{\ln\left|\ln\left(\frac{1}{2}\right)\right|-\overbrace{\ln|\underbrace{\ln(\varepsilon)}_{+\infty}|}^{+\infty}\right\}=-\infty
\end{gather*}
L'integrale improprio diverge a $-\infty$.
%% TODO add:
% Integrali impropri notevoli:
% \begin{gather*}
%     \displaystyle\int_0^{1/2}\frac{1}{x^\alpha\ln^\beta x}\mathrm{d}x
% \end{gather*}
% Qualunque sia il valore di $\beta$, se $\alpha<1$, l'integrale converge. 


Studiare la convergenza del seguente integrale:
\begin{gather*}
    \displaystyle\int_0^1\frac{e^x}{\sqrt{1-\cos x}}\mathrm{d}x
\end{gather*}
Presenta un punto di singolarità per $x\to0^+$, si considerano le equivalenze asintotiche per $x\to0^+$:, per il limite notevole $1-\cos x/x^2$:
\begin{gather*}
    \displaystyle\frac{e^x}{\sqrt{1-\cos x}}\sim\frac{1}{\sqrt{\frac{1}{2}x^2}}=\sqrt{2}\frac{1}{x}
\end{gather*}
L'integrale improprio di questa funzione diverge, avendo un esponente maggiore o uguale ad uno, quindi la funzione integranda di partenza non è integrabile in senso improprio. 
Un modo alternativo consiste nello sfruttare l'espansione in serie del coseno, per $x\to0^+$, mediante la serie di Maclaurin
\begin{gather*}
    \cos x=1-\displaystyle\frac{x^2}{2}+o[x^2]
\end{gather*}
Sostituendo quest'approssimazione nella funzione integranda si ha:
\begin{gather*}
    1-\cos x=\displaystyle\frac{x^2}{2}+o[x^2]\\
    \displaystyle\frac{e^x}{\sqrt{1-\cos x}}=\frac{e^x}{\sqrt{\frac{x^2}{2}+o[x^2]}}
\end{gather*}

Considerando il seguente integrale improprio, bisogna utilizzare la formula di Maclaurin di ordine quattro:
\begin{gather*}
    \displaystyle\int_0^1\frac{e^x}{\sqrt[3]{1-\frac{x^2}{2}-\cos x}}\mathrm{d}{x}\\
    \cos x=1-\displaystyle\frac{x^2}{2}+\frac{x^4}{4!}+o[x^4]\implies
    \displaystyle1-\frac{x^2}{2}-\cos x=-\frac{x^4}{4!}+o[x^4]\\
    \displaystyle\frac{1-\frac{x^2}{2}-\cos x}{-\frac{x^4}{4!}}=1+o[1]\\
    \displaystyle\frac{e^x}{\sqrt[3]{1-\frac{x^2}{2}-\cos x}}\sim-\frac{\sqrt[3]{4!}}{\sqrt[3]{x^4}}
\end{gather*}
L'integrale di questa funzione diverge poiché $\alpha>1$. 


Valutare la convergenza del seguente integrale:
\begin{gather*}
    \displaystyle\int_0^1\frac{\sin x+\cos x}{\sqrt[5]{1-x^3}}\mathrm{d}x
\end{gather*}
La funzione integranda presenta un punto di singolarità per $x=1$, nell'intorno sinistro di uno la funzione si mantiene positiva, quindi si può applicare il criterio del confronto asintotico, il numeratore non crea problemi, essendo $\sin x+\cos x\sim \sin1+\cos1$, si ha per prodotto notevole della differenza di due cubi:
\begin{gather*}
    \displaystyle\frac{\sin x+\cos x}{\sqrt[5]{1-x^3}}\sim\frac{\sin1+\cos1}{\sqrt[5]{1-x}\sqrt[5]{1+x+x^2}}\sim
    \frac{\sin 1+\cos 1}{\sqrt[5]{3}}\frac{1}{(1-x)^{1/5}}
\end{gather*}
L'integrale di questa funzione converge perché $\alpha<1$. 


Studiare la convergenza del seguente integrale improprio:
\begin{gather*}
    \displaystyle\int_0^1\frac{1}{e^x-\cos x}
\end{gather*}
Si considera l'espansione mediante la serie di Maclaurin per $x\to0^+$, del denominatore:
\begin{gather*}
    e^x=1+x+o[x]\\
    \cos x=1+o[x]\\
    e^x-\cos x=\cancel1+x+{o[x]}-\cancel1-{o[x]}=x+o[x]\implies e^x-\cos x\sim x
    \displaystyle\frac{1}{e^x-\cos x}\sim\frac{1}{x}
\end{gather*}
L'integrale di questa funzione non converge, quindi la funzione di partenza non è integrabile in senso improprio. 


Considerare il seguente integrale improprio:
\begin{gather*}
    \displaystyle\int_0^1\frac{1-e^{-x}}{x\sqrt{x}}\mathrm{d}x
\end{gather*}
La funzione ha un punto critico per $x=0$, e si può usare il criterio del confronto asintotico, essendo definitamente positiva per $x\to0^+$. Al denominatore è già presente la funzione campione. Si considera l'espansione mediante la formula di Maclaurin dell'esponenziale:
\begin{gather*}
    e^{-x}=1-x+o[x]\implies1-e^{-x}=x+o[x]\\
    \displaystyle\frac{1-e^{-x}}{x^{3/2}}\sim\frac{x}{x^{3/2}}=\frac{1}{x^{1/2}}
\end{gather*}
La funzione ottenuta è integrabile in modo improprio, quindi l'integrale di partenza converge. 



Studiare la convergenza del seguente integrale improprio:
\begin{gather*}
    \displaystyle\int_0^1\sin\left(\frac{1}{x}\right)e^{-\frac{1}{x}}x^{-2}\mathrm{d}x
\end{gather*}
La funzione integranda ha un punto di singolarità per $x=0$. La funzione non si mantiene positiva nell'intorno di zero, poiché il seno continua ad oscillare a frequenza sempre maggiore per $x\to0^+$. Quindi non si può utilizzare il criterio del confronto asintotico, quindi si studia la convergenza assoluta:
\begin{gather*}
    \displaystyle\int_0^1\left|\sin\left(\frac{1}{x}\right)e^{-\frac{1}{x}}x^{-2}\right|\mathrm{d}x
\end{gather*}
Si può applicare su questa funzione integranda il criterio del confronto:
\begin{gather*}
    \left|\sin\left(\frac{1}{x}\right)e^{-\frac{1}{x}}x^{-2}\right|\leq e^{-\frac{1}{x}}x^{-2}
\end{gather*}
Si determina una funzione asintoticamente equivalente a questa funzione maggiorante:
\begin{gather*}
    \lim_{x\to0^+}\displaystyle\frac{e^{-\frac{1}{x}}x^{-2}}{x^{1/2}}
\end{gather*}
Se questo limite tende a zero, allora la funzione maggiorante converge. 
\begin{gather*}
    \lim_{x\to0^+}\displaystyle\frac{x^{-5/2}}{e^{1/x}}
\end{gather*}
Il denominatore è un infinito di ordine superiore, quindi il limite tende a zero. Un corollario del teorema del confronto asintotico afferma che se il limite della funzione integranda fratto un'altra funzione $g(x)$ è uguale a zero, e questa funzione ammette un integrale improprio convergente, allora lo ammette anche la funzione integranda. Quindi converge anche l'integrale improprio di partenza. 
Questo può valere per qualsiasi funzione $g(x)=x^\alpha$, con $\alpha<1$. 

\section{Esercitazione 4/4/25}

Studiare la convergenza di questo integrale improprio:
\begin{gather*}
    \displaystyle\int_1^{+\infty}\frac{\df x}{x\sqrt{1+\sqrt{x}}}
\end{gather*}
La funzione è positiva in tutto l'intervallo, ed è un integrale improprio di prima specie. Considerando la funzione integranda si mette in evidenza la radice di $x$, per $x\to+\infty$:
\begin{gather*}
    \displaystyle\frac{1}{x\sqrt{1+\sqrt{x}}}=\frac{1}{x\sqrt{\sqrt{x}\left(\frac{1}{\sqrt{x}}+1\right)}}=
    \frac{1}{x\sqrt{\sqrt{x}}\cdot\sqrt{\frac{1}{\sqrt{x}}+1}}\\
    \displaystyle\sqrt{\frac{1}{\sqrt{x}+1}}\to1\implies
    \frac{1}{x\sqrt{\sqrt{x}}\cdot\sqrt{\frac{1}{\sqrt{x}}+1}}\sim
    \frac{1}{x\sqrt{\sqrt{x}}}=\frac{1}{x^{5/4}}
\end{gather*}
Questa funzione è integrabile in modo improprio nell'intervallo $[1,+\infty)$, quindi per il criterio del confronto asintotico lo è anche la funzione integranda di partenza. 


Studiare la convergenza di questo integrale:
\begin{gather*}
    \intinf{\frac{1}{(1+x^2)^2}}{x}
    =\int_{\mathbb{R}}\frac{1}{(1+x^2)^2}\df x
\end{gather*}
Si applica il teorema del confronto asintotico, essendo monotona positiva, per $x\to+\infty$
\begin{gather*}
    \frac{1}{(1+x^2)^2}=
    \frac{1}{1+2x^2+x^4}\sim\frac{1}{x^4}
\end{gather*}
essendo la funzione pari si può calcolare analogamente come:
\begin{gather*}
    \intinf{\frac{1}{(1+x^2)^2}}{x}=
    \intpinf{\frac{1}{(1+x^2)^2}}{x}
\end{gather*}
In questo modo bisogna calcolare un limite in meno, solo per $h\to+\infty$. 




Verificare la convergenza o meno di questo integrale:
\begin{gather*}
    \displaystyle\int_{1/2}^{+\infty}\frac{1}{x^2\sqrt{4x^2-1}}\df x
\end{gather*}
Questo è un integrale improprio di prima e di seconda specie, avendo un punto critico in $x=1/2$, si sfrutta la proprietà additiva degli integrali, scegliendo un punto intermedio $a>1/2$:
\begin{gather*}
    \displaystyle\int_{1/2}^{+\infty}\frac{1}{x^2\sqrt{4x^2-1}}\df x=
    \overbrace{\int_{1/2}^{a}\frac{1}{x^2\sqrt{4x^2-1}}\df x}^{I_1}+
    \underbrace{\int_{a}^{+\infty}\frac{1}{x^2\sqrt{4x^2-1}}\df x}_{I_2}
\end{gather*}
Entrambi questi integrali devono converge, affinché l'integrale di partenza converga. 
Si considera la funzione integranda, espressa mediante il prodotto notevole della somma di quadrati. Per $x\to1/2$ si ha:
\begin{gather*}
    \displaystyle\frac{1}{x^2\sqrt{4x^2-1}}=\frac{1}{x^2\sqrt{2x-1}\sqrt{2x+1}}\sim\frac{1}{\frac{1}{4}\sqrt{2\left(x-\frac{1}{2}\right)}\sqrt{2}}=\frac{1}{\frac{1}{2}\left(x-\frac{1}{2}\right)^{1/2}}
\end{gather*}
Questa funzione è integrabile in modo improprio di seconda specie, avendo un esponente $\alpha<1$, quindi $I_1$ converge. 
Si considera ora la funzione integranda di $I_2$ per $x\to+\infty$:
\begin{gather*}
    \displaystyle\frac{1}{x^2\sqrt{4x^2-1}}=\frac{1}{x^2x\sqrt{4-\frac{1}{x^2}}}\sim\frac{1}{2x^3}
\end{gather*}
Poiché questa funzione ha un esponente maggiore di uno $\alpha>1$ questa funzione è integrabile in modo improprio di prima specie, quindi l'integrale $I_2$ converge. 


Studiare la convergenza del seguente integrale improprio:
\begin{gather*}
    \displaystyle\int_1^{+\infty}\frac{\ln x}{x}\df x
\end{gather*}
Essendo una funzione logaritmica non confrontabile, si utilizza la definizione per calcolarlo:
\begin{gather*}
    \displaystyle\int_1^{+\infty}\frac{\ln x}{x}\df x=
    \lim_{h\to\infty}\int_1^{h}\ln x\df(\ln x)=
    \lim_{h\to\infty}\frac{\ln^2x}{2}\bigg|_1^h=+\infty
\end{gather*}
Quindi l'integrale non converge. Questo integrale è un caso particolare di questo integrale notevole:
\begin{gather*}
    \displaystyle\int_{a}^{+\infty}\frac{\df x}{x^\alpha(\ln x)^\beta}
\end{gather*}
Questo integrale è dimostrabile converge se $\alpha>1$ e $\forall\beta\in\mathbb{R}$. Oppure converge sempre se $\alpha=1$ e $\beta>1$. Mentre l'integrale diverge a $+\infty$ se $\alpha<1$, qualunque sia $\beta$ reale: $\forall\beta\in\mathbb{R}$, oppure se $\alpha=1$ e $\beta\leq1$. 



Valutare la convergenza del seguente integrale:
\begin{gather*}
    \displaystyle\int_1^{+\infty}\frac{\ln\left(1+\frac{1}{x}\right)}{\sqrt{x^2-1}}\df x
\end{gather*}
Essendo sia di prima che di seconda specie si divide tramite la proprietà additiva degli integrali, con $a>1$:
\begin{gather*}
    \displaystyle\int_1^a\frac{\ln\left(1+\frac{1}{x}\right)}{\sqrt{x^2-1}}\df x
\end{gather*}
Si considera la funzione integranda per $x\to1^+$:
\begin{gather*}
    \displaystyle\frac{\ln\left(1+\frac{1}{x}\right)}{\sqrt{x^2-1}}\sim\frac{\ln2}{\sqrt{x-1}\sqrt{x+1}}\sim\frac{\ln2}{\sqrt2(x-1)^{1/2}}
\end{gather*}
Questa funzione è integrabile in modo improprio di seconda specie, si considera ora l'altro caso:
\begin{gather*}
    \displaystyle\int_a^{+\infty}\frac{\ln\left(1+\frac{1}{x}\right)}{\sqrt{x^2-1}}\df x
\end{gather*}
Per $x\to+\infty$ la funzione integranda diventa:
\begin{gather*}
    \frac{\ln\left(1+\frac{1}{x}\right)}{\sqrt{x^2-1}}\df x\sim\frac{\frac{1}{x}}{x}=\frac{1}{x^2}
\end{gather*}
Essendo $\alpha>1$ l'integrale per $[a,+\infty)$ converge. 



Studiare la convergenza del seguente integrale improprio:
\begin{gather*}
    \intpinf{\frac{\sin\left(\frac{1}{x^2}\right)}{\ln\left(1+\sqrt{x}\right)}}{x}
\end{gather*}
Si divide in due integrali in un punto intermedio $a>0$ essendo improprio sia di prima che di secona specie:
\begin{gather*}
    \intpinf{\frac{\sin\left(\frac{1}{x^2}\right)}{\ln\left(1+\sqrt{x}\right)}}{x}=
    \overbrace{\int_0^a\frac{\sin\left(\frac{1}{x^2}\right)}{\ln\left(1+\sqrt{x}\right)}\df x}^{I_1}+
    \underbrace{\int_a^{+\infty}\frac{\sin\left(\frac{1}{x^2}\right)}{\ln\left(1+\sqrt{x}\right)}\df x}_{I_2}
\end{gather*}
Si considera l'integrale $I_1$, ma non esistono criteri per poter valutare la sua convergenza. L'unico strumento a disposizione è studiare la convergenza assoluta:
\begin{gather*}
    \displaystyle\int_0^a\left|\frac{\sin\left(\frac{1}{x^2}\right)}{\ln\left(1+\sqrt{x}\right)}\right|\df x=\int_0^a\frac{\left|\sin\left(\frac{1}{x^2}\right)\right|}{\ln\left(1+\sqrt{x}\right)}\df x
\end{gather*}
Si considera la funzione maggiorante, per $x\to0^+$:
\begin{gather*}
    \frac{|\sin\left(\frac{1}{x^2}\right)|}{\ln\left(1+\sqrt{x}\right)}\leq\frac{1}{\ln(1+\sqrt{x})}\sim\frac{1}{\sqrt{x}}
\end{gather*}
La funzione è integrabile in senso improprio avendo $\alpha<1$, $I_1$ converge in senso assoluto quindi anche semplicemente. Si analizza ora $I_2$, per $x\to+\infty$, la sua funzione integranda diventa:
\begin{gather*}
    \displaystyle\frac{\sin\overbrace{\frac{1}{x^2}}^{\to0}}{\ln\left(1+\sqrt{x}\right)}\sim
    \frac{\frac{1}{x^2}}{\ln\left(1+\sqrt{x}\right)}<\frac{1}{x^2}
\end{gather*}
Per il teorema del confronto l'integrale improprio di prima specie di $x^{-2}$ converge, quindi converge per il criterio del confronto asintotico $I_2$. 



Valutare la convergenza del seguente integrale improprio di prima specie:
\begin{gather*}
    \displaystyle\int_1^{+\infty}\frac{\df x}{x+\sin^2x}
\end{gather*}
Si maggiora la funzione integranda:
\begin{gather*}
    x+\sin^2x\leq x+1\implies\displaystyle\frac{1}{x+\sin^2x}\geq\frac{1}{x+1}\sim\frac{1}{x}
\end{gather*}
Per $x\to+\infty$ l'integrale notevole improprio che minora la funzione integranda diverge, quindi anche l'integrale diverge a $+\infty$. 



Studiare la convergenza del seguente integrale improprio:
\begin{gather*}
    \displaystyle\int_1^{+\infty}\frac{x+\sqrt{x+1}}{x^2+2\sqrt[5]{x^4+1}}\df x
\end{gather*}
Si raccoglie al numeratore ed al denominatore il fattore $x$ di esponente maggiore:
\begin{gather*}
    \displaystyle\frac{x+\sqrt{x+1}}{x^2+2\sqrt[5]{x^4+1}}=
    \frac{x+\sqrt{x}\sqrt{\frac{1}{x}+1}}{x^2+2\sqrt[5]{x^4\left(1+\frac{1}{x^4}\right)}}=
    \frac{x+\sqrt{x}\sqrt{\frac{1}{x}+1}}{x^2+2x^{4/5}\sqrt[5]{\left(1+\frac{1}{x^4}\right)}}\sim\frac{x}{x^2}=\frac{1}{x}
\end{gather*}
Avendo $\alpha=1$, la funzione non è integrabile in senso improprio, quindi anche l'integrale di partenza non converge. 



Studiare la convergenza del seguente integrale improprio, per $\alpha\in\mathbb{R}$: 
\begin{gather*}
    \intpinf{\frac{\sqrt{x}}{(1-\cos x+x^2)^{\alpha-1}(x+1)^2}}{x}
\end{gather*}
È un integrale improprio di prima specie e di seconda specie avendo un punto critico in $x=0$. Si considera un punto intermedio $a>0$:
\begin{gather*}
    \displaystyle\int_0^a\frac{\sqrt{x}}{(1-\cos x+x^2)^{\alpha-1}(x+1)^2}\df x
\end{gather*}
Per determinare una funzione asintoticamente equivalente alla funzione integranda per $x\to0^+$ si utilizza la formula di Maclaurin per il coseno:
\begin{gather*}
    \cos=1-\displaystyle\frac{x^2}{2}+o[x^2]\implies
    1-\cos x+x^2=\frac{3}{2}x^2+o[x^2]\\
    1-\cos x+x^2\sim\displaystyle\frac{3}{2}x^2\\
    \displaystyle\frac{\sqrt{x}}{(1-\cos x+x^2)^{\alpha-1}(x+1)^2}
    \sim\frac{\sqrt{x}}{(\frac{3}{2}x^2)^{\alpha-1}\cdot(1)}=\left(\frac{2}{3}\right)^{\alpha-1}\frac{x^{1/2}}{x^{2\alpha-2}}=\left(\frac{2}{3}\right)^{\alpha-1}\frac{1}{x^{2\alpha-5/2}}
\end{gather*}
La funzione converge se l'esponente è minore di uno, essendo un integrale di seconda specie: $2\alpha-5/2<1$:
\begin{gather*}
    2\alpha-\displaystyle\frac{5}{2}<1\implies\alpha<\frac{7}{4}
\end{gather*}

Si considera ora l'integrale di prima specie:
\begin{gather*}
    \displaystyle\frac{\sqrt{x}}{(1-\cos x+x^2)^{\alpha-1}(x+1)^2}\sim
    \frac{x^{1/2}}{x^{2\alpha-2}\cdot x^2}=\frac{1}{x^{2\alpha-1/2}}
\end{gather*}
L'integrale converge se l'esponente è maggiore di uno, essendo un integrale di prima specie:
\begin{gather*}
    2\alpha-\displaystyle\frac{1}{2}>1\implies\alpha>\frac{3}{4}
\end{gather*}

L'integrale converge se $\alpha\in[3/4,7/4]$. 



%% TODO foto fine lezione 4/4/25
% Studiare la convergenza del seguente integrale  per $\alpha\in\mathbb{R}$:
% \begin{gather*}
%     \displaystyle\int_1^{+\infty}\frac{\df x}{}
% \end{gather*}

\clearpage

\section{Esercizi 8/4/25}

Calcolare la serie di Fourier di questa funzione:
\begin{gather*}
    f(x)=x\,\mbox{se}\,x\in[-\pi,\pi)\\
    f(x+2\pi)=f(x)\, \forall x\in\mathbb{R}
\end{gather*}

%% TODO img funzione 

La funzione è dispari, nell'intervallo aperto, allora $a_k=0\,\forall k\in\mathbb{N}$, la rispettiva serie di Fourier si esprime come:
\begin{gather*}
    \displaystyle\sum_{k=1}^\infty b_k\sin\left(kx\right)=
    \displaystyle\sum_{k=1}^\infty\left[\frac{2}{\pi}\int_0^{\pi}x\sin\left(kx\right)\df x\right]\sin(kx)
\end{gather*}
Si risolve l'integrale per parti, e si ottiene un coefficiente $b_k$:
\begin{gather*}
    b_k=\frac{2}{\pi}\int_0^{\pi}x\sin\left(kx\right)\df x=\left[-\frac{2}{k\pi}x\cos(kx)+\frac{2}{k^2\pi}\bcancel{\sin(kx)}\bigg|_0^\pi\right.=-\frac{2}{k}(-1)^k=(-1)^{k+1}\frac{2}{k}\\
    \displaystyle\sum_{k=1}^\infty (-1)^{k+1}\frac{2}{k}\sin\left(kx\right)=
    2\sum_{k=1}^\infty \frac{(-1)^{k+1}}{k}\sin\left(kx\right)
\end{gather*}
%% SVOLGIMENTO PER PARTI
% D       I
% x       sin(kx)
% 1       -cos(kx)/k
% 0       -sin(kx)/k^2
% --------------------
% -xcos(kx)/k+sin(kx)/k^2
%% TODO add to teoria. 
La formula $\cos(k\pi)=(-1)^k$ è molto ricorrente nell'ambito delle serie di Fourier. Sostituendo $x=\pi/4$ si può ottenere l'identità:
\begin{gather*}
    2\sum_{k=1}^\infty \frac{(-1)^{k+1}}{k}\sin\left(k\frac{\pi}{2}\right)=f\left(\frac{\pi}{2}\right)=\frac{\pi}{2}\\
    \displaystyle\frac{\pi}{4}=\overbrace{\sin\left(\frac{\pi}{2}\right)}^1-\frac{1}{2}\underbrace{\sin(\pi)}_{0}+\frac{1}{3}\overbrace{\sin\left(\frac{3}{2}\pi\right)}^{-1}-\frac{1}{4}\underbrace{\sin(2\pi)}_0+\frac{1}{5}\overbrace{\sin\left(\frac{5}{2}\pi\right)}^1+\cdots\\
    \displaystyle1-\frac{1}{4}+\frac{1}{5}-\frac{1}{7}+\frac{1}{9}-\cdots=\frac{\pi}{4}=\displaystyle\sum_{k=0}^\infty\frac{(-1)^k}{2k+1}
\end{gather*}

%% TODO esercizi da foto 8/4/25 ~15:45:xx

\section{Esercizi 11/4/25}

Calcolare la serie di Fourier della seguente funzione:
\begin{gather*}
    f(x)=|x|\,\mbox{ se } x\in[-\pi,\pi)
\end{gather*}
Inoltre $\forall x\in\mathbb{R}$ si ha $f(x+2\pi)=f(x)$. Inoltre calcolare la somma:
%% TODO grafico funzione
\begin{gather*}
    \displaystyle\sum_{k=0}^\infty\frac{1}{(2k+1)^2}
\end{gather*}
La serie di Fourier permette di calcolare più facilmente alcune somme. %% TODO add teoria

Inoltre la funzione è pari, per cui la sua serie di Fourier si può esprimere come:
\begin{gather*}
    f(x)=\displaystyle\frac{a_0}{2}+\sum_{k=1}^\infty a_k\cos\left(\frac{2k\pi}{T}x\right)+b_k\sin\left(\frac{2k\pi}{T}x\right)\\
    b_k=\displaystyle\frac{2}{T}\int_{-T/2}^{T/2}f(x)\sin\left(\frac{2k\pi}{T}x\right)\df x=0
\end{gather*}
La serie di Fourier diventa:
\begin{gather*}
    f(x)=\displaystyle\frac{a_0}{2}+\sum_{k=1}^\infty a_k\cos\left(\frac{2k\pi}{T}x\right)\\
    a_k=\displaystyle\frac{2}{T}\int_{-T/2}^{T/2}f(x)\cos\left(\frac{2k\pi}{T}x\right)\df x=\displaystyle\frac{4}{T}\int_{0}^{T/2}f(x)\cos\left(\frac{2k\pi}{T}x\right)\df x=\frac{2}{\pi}\int_0^{\pi}|x|\cos(kx)\df x\\
    a_0=\displaystyle\frac{2}{\pi}\int_{0}^\pi x\cancelto{1}{\cos(0\cdot x)}\df x=\frac{\cancel{2}}{\cancel{\pi}}\frac{\pi^{\cancel{2}}}{\cancel{2}}=\pi\\
    \displaystyle\frac{2}{\pi}\int_0^{\pi}|x|\cos(kx)\df x=\frac{2}{\pi}\int_0^{\pi}x\cos(kx)\df x=\left[\frac{2}{k\pi}x\sin(kx)+\frac{2}{k^2\pi}\cos(kx)\right.\bigg|_0^\pi\\
    a_k=\left[\frac{2}{k\pi}\cancelto{0}{x\sin(kx)}+\frac{2}{k^2\pi}\cos(kx)\right.\bigg|_0^\pi=\frac{2}{k^2\pi}(-1)^k-\frac{2}{k^2\pi}=\frac{2}{k^2\pi}\left((-1)^{k}-1\right)
\end{gather*}
%   D     I
% + x     cos kx
% - 1     sin kx/k
% + 0     -cos x/k^2

La serie di Fourier è quindi:
\begin{gather*}
    f(x)=\displaystyle\frac{\pi}{2}+\sum_{k=1}^\infty\frac{2}{k^2\pi}\left((-1)^{k}-1\right)\cos(kx)=\frac{\pi}{2}+\frac{2}{\pi}\sum_{k=1}^\infty\frac{(-1)^{k}-1}{k^2}\cos(kx)\\
    f(x=0)=0=\displaystyle\frac{\pi}{2}+\frac{2}{\pi}\sum_{k=1}^\infty\frac{(-1)^{k}-1}{k^2}\\
    \displaystyle\sum_{k=1}^\infty\frac{(-1)^{k}-1}{k^2}=-\frac{\pi^2}{4}=-2+0-\frac{2}{3^2}+0-\frac{2}{5^2}+0-\cdots\\
    -2\left(1+\displaystyle\frac{1}{3^2}+\frac{1}{5^2}+\frac{1}{7^2}+\cdots\right)=-2\sum_{k=0}^\infty\frac{1}{(2k+1)^2}=-\frac{\pi^2}{4}\\
    \displaystyle\sum_{k=0}^\infty\frac{1}{(2k+1)^2}=\frac{\pi^2}{8}
\end{gather*}
Si è individuata la somma della serie proposta all'inizio è convergente e finita. 



Calcolare la serie di Fourier della seguente funzione:
\begin{gather*}
    f(x)=x^2\,\mbox{ se } x\in[-\pi,\pi)
\end{gather*}
Inoltre $\forall x\in\mathbb{R}$ si ha $f(x+2\pi)=f(x)$. Inoltre calcolare la somma:
%% TODO grafico funzione
\begin{gather*}
    \displaystyle\sum_{k=0}^\infty\frac{1}{k^2}%=\frac{\pi^2}{6}
\end{gather*}
Questa funzione è pari quindi la sua serie di Fourier diventa:
\begin{gather*}
    f(x)=\displaystyle\frac{a_0}{2}+\sum_{k=1}^\infty a_k\cos(kx)\\
    a_0=\displaystyle\frac{2}{\pi}\int_0^\pi x^2\df x=\frac{2}{\cancel{\pi}}\frac{\pi^{\cancel{3}}}{3}=\frac{2\pi^2}{3}\\
    a_k=\displaystyle\frac{2}{\pi}\int_0^\pi x^2\cos(kx)\df x=\frac{2}{\pi}\left[\frac{1}{k}x^2\sin(kx)+\frac{2}{k^2}x\cos(kx)-\frac{2}{k^3}\sin(kx)\right.\bigg|_0^\pi\\
    a_k=\frac{2}{\pi}\left[\frac{1}{k}\cancel{x^2\sin(kx)}+\frac{2}{k^2}x\cos(kx)-\frac{2}{k^3}\bcancel{\sin(kx)}\right.\bigg|_0^\pi=\frac{4}{k^2\pi}x\cos(kx)\bigg|_0^\pi=\frac{4}{k^2\cancel{\pi}}\cancel{\pi}(-1)^k\\
    f(x)=\frac{\pi^3}{3}+\sum_{k=0}^\infty\frac{4}{k^2}(-1)^k\cos(kx)
\end{gather*}
%   D       I
% + x^2     cos(kx)
% - 2x      sin(kx)/k
% + 2       -cos(kx)/k^2
% - 0       -sin(kx)/k^3

Per arrivare alla serie iniziale si sceglie $x=\pi$:
\begin{gather*}
    f(\pi)=\pi^2=\frac{\pi^2}{3}+4\sum_{k=0}^\infty\frac{(-1)^k}{k^2}(-1)^k\\
    \sum_{k=0}^\infty\frac{1}{k^2}=\frac{2\pi^2}{3}\frac{1}{4}=\frac{\pi^2}{6}
\end{gather*}

\section{Esercizi 6/5/25}

Considerare la seguente funzione:
\begin{gather*}
    f(x,y)=\begin{cases}
        \displaystyle\frac{\strut x^4+y^2e^x}{\strut\sqrt{x^2+y^2}}&(x,y)\neq(0,0)\\
        0&(x,y)=(0,0)
    \end{cases}
\end{gather*}
Studiare la continuità e la derivabilità parziale rispetto alla $x$ e la $y$ in $(0,0)$. 

Si considera la continuità della funzione nell'origine, quindi si calcola il valore del limite:
\begin{gather*}
    \lim_{(x,y)\to(0,0)}f(x,y)=\lim_{(x,y)\to(0,0)}\displaystyle\frac{x^4+y^2e^x}{\sqrt{x^2+y^2}}
\end{gather*}
Si effettua una trasformazione in coordinate polari:
\begin{gather*}
    \lim_{(x,y)\to(0,0)}\displaystyle\frac{x^4+y^2e^x}{\sqrt{x^2+y^2}}=
    \lim_{\rho\to0}\frac{\rho^4\cos^4\theta+\rho^2\sin^2\theta e^{\rho\cos\theta}}{\sqrt{\rho^2(\cos^2\theta+\sin^2\theta)}}=
    \lim_{\rho\to0}{\rho^3\cos^4\theta+\rho\sin^2\theta e^{\rho\cos\theta}}    
\end{gather*}
Si considera la formula di Maclaurin per l'esponenziale:
\begin{gather*}
    e^{\rho\cos\theta}=1+\rho\cos\theta+o[\rho]\\
    \lim_{\rho\to0}{\rho^3\cos^4\theta+\rho\sin^2\theta e^{\rho\cos\theta}}=
    \lim_{\rho\to0}{\rho^3\cos^4\theta+\rho\sin^2\theta (1+\rho\cos\theta+o[\rho])}\\
    \lim_{\rho\to0}\rho^3\cos^4\theta+\rho\sin^2\theta+\rho^2\sin\theta\cos\theta+o[\rho^2]    
\end{gather*}
Tutti i termini di questo limite tendono a zero, quindi anche il limite complessivo deve tendere a zero. La funzione è quindi continua nell'origine. Tutti gli addendi possono essere analizzai con il teorema del confronto. 

Per determinare la derivabilità si considera un incremento $h$ lungo l'asse $x$ ed un incremento $k$ lungo l'asse $y$ e si calcola il limite del loro rapporto incrementale:
\begin{gather*}
    \lim_{h\to0}\frac{f(h,0)-f(0,0)}{h}=
    \lim_{h\to0}\frac{\frac{h^4}{\sqrt{h^2}}-0}{h}=
    \lim_{h\to0}\frac{h^{\cancelto{2}{3}}}{\cancel{h}}=0\\
    \lim_{k\to0}\frac{f(0,k)-f(0,0)}{k}=
    \lim_{k\to0}\frac{\frac{k^2e^0}{\sqrt{k^2}}-0}{k}=
    \lim_{k\to0}\frac{k}{|k|}=\nexists
\end{gather*}
La funzione è derivabile in modo parziale rispetto a $x$ nell'origine, ma non rispetto a $y$, ed ha valore
\begin{gather*}
    \displaystyle\frac{\partial f}{\partial x}(0,0)=0
\end{gather*}



\end{document}
