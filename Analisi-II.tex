\documentclass{article}

\usepackage{cancel}
\usepackage{amsmath,amssymb}
\usepackage[includehead,nomarginpar]{geometry}
\usepackage{graphicx}
\usepackage{amsfonts} 
\usepackage{verbatim}
\usepackage{mathrsfs}  
\usepackage{lmodern}
\usepackage{braket}
\usepackage{bookmark}
\usepackage{fancyhdr}
\usepackage{romanbarpagenumber}
%\usepackage{minted}
%\usepackage{subfig}
\usepackage[italian]{babel}
\usepackage{float}
%\usepackage{wrapfig}
%\usepackage[export]{adjustbox}
\usepackage{contour}
\usepackage[normalem]{ulem}
\allowdisplaybreaks

\setlength{\headheight}{12.0pt}
\addtolength{\topmargin}{-12.0pt}
\graphicspath{ {./Immagini/} }

%% TODO add metadata
\hypersetup{
    colorlinks=true,
    linkcolor=black,
    pdftitle={Appunti di Analisi II},
    pdfauthor={Giacomo Sturm},
    pdfsubject={Analisi II},
    pdfkeywords={}
}

\newsavebox{\tempbox} %{\raisebox{\dimexpr.5\ht\tempbox-.5\height\relax}}

%% TODO aggiungere comandi funzioni

\makeatother
\renewcommand{\contentsname}{Indice}
\numberwithin{equation}{subsection}
\newcommand{\tageq}{\tag{\stepcounter{equation}\theequation}}
\AtBeginDocument{%
    \renewcommand{\figurename}{Fig.}
}
\renewcommand{\ULdepth}{1.8pt}
\contourlength{0.6pt}
\newcommand{\myuline}[1]{%
    \uline{\phantom{#1}}%
    \llap{\contour{white}{#1}}%
}
\fancypagestyle{link}{\fancyhf{}\renewcommand{\headrulewidth}{0pt}\fancyfoot[C]{Sorgente del file \LaTeX disponibile al seguente link: \url{https://github.com/00Darxk/Analisi-II/}}}

\begin{document}

\title{%
    \textbf{Analisi II}  \\ 
    \large Appunti delle Lezioni di Analisi II \\
    \textit{Anno Accademico: 2024/25}}
\author{\textit{Giacomo Sturm}}
\date{\textit{Dipartimento di Ingegneria Industriale, Elettronica e Meccanica \\
Università degli Studi ``Roma Tre"}} 

\maketitle
\thispagestyle{link}

\clearpage


\pagestyle{fancy}
\fancyhead{}\fancyfoot{}
\fancyhead[C]{\textit{Analisi II - Università degli Studi ``Roma Tre"}}
\fancyfoot[C]{\thepage}
\pagenumbering{Roman}

\tableofcontents

\clearpage
\pagenumbering{arabic}

\section{Serie Numerica}

Una serie numerica $\{a_n\}$ è una \textit{particolare} successione numerica; 
è una particolare funzione che ha come dominio l'insieme dei numeri naturali $\mathbb{N}$ e come codominio un sottoinsieme $\mathbb{C}$ dell'asse dei numeri reali $\mathbb{R}$. 
\begin{equation}
    \{a_n\}:\mathbb{N}\rightarrow\mathbb{C}\in\mathbb{R}
\end{equation}

Quindi è di interesse studiare il comportamento della serie numerica quando la variabile indipendente $n$ tende all'infinito. Data una successione numerica, si interessa quindi:
\begin{equation}
    \lim_{n\to\infty}a_n
\end{equation}

Si generalizza nel limite di una funzione quando si sostituisce alla variabile $n$ una variabile indipendente reale $x\in\mathbb{R}$. 

Una successione può convergere ad un numero reale, può divergere positivamente o negativamente, in questi due casi è regolare, quindi ha un limite definito o indefinito. Altrimenti è possibile che il limite non esiste, quindi la sequenza non si stabilisce su nessun numero reale ed oscilla per $n$ tendente ad $\infty$. 

\begin{equation}
    \lim_{n\to\infty}a_n=\begin{cases}
        i\in\mathbb{R}\\
        \pm\infty\\
        \nexists
    \end{cases}
\end{equation}

Si costruisce una serie numerica $\{s_n\}$, partendo da una sequenza nota $\{a_k\}$, si può costruire definendo per ogni valore $n$ il valore di $s_n$ tale sia pari alla somma dei primi $a_n$ elementi della sequenza:
\begin{equation}
    \forall n\in\mathbb{N}:\,
    s_n:=a_1+\cdots+a_n
\end{equation}

$s_n$ viene chiamata somma parziale della serie $\{s_n\}$ e viene definita con il seguente simbolo:
\begin{equation}
    s_n:=\displaystyle\sum_{k=1}^\infty a_k
\end{equation}
La serie numerica è la successione delle somme parziali che costruisce sulla successione $\{a_k\}$. 

Una serie numerica si dice convergente se il limite della successione delle somme parziali, costruite a partire dalla successione $a_k$, è pari ad un numero $s$:
\begin{equation}
    \{s_n\}: \mbox{converge se}\,\lim_{n\to\infty}s_n=s\in\mathbb{R}
\end{equation}

Si dice divergente se il limite della successione delle somme parziali tende ad infinito:
\begin{equation}
    \{s_n\}: \mbox{diverge se}\,\lim_{n\to\infty}s_n=\pm\infty
\end{equation}

Se converge o diverge la serie numerica si dice regolare, se invece studiando il comportamento delle somme parziali per $n\to\infty$, il limite della somma non è definito, quindi la serie si dice irregolare:
\begin{equation}
    \{s_n\}: \mbox{irregolare se}\,\lim_{n\to\infty}s_n=\nexists
\end{equation}

Studiare il comportamento della serie vuol dire studiare il comportamento delle somme parziali. La somma parziali si avrà sempre in forma aperta, se fosse possibile esprimere in forma chiusa la forma parziale, allora il limite è di facile calcolo. 

%% ESEMPIO

Considerando il seguente limite:
\begin{equation*}
    \lim_{n\to\infty}(1+2+\cdots+n)
\end{equation*}
Esiste una funzione per esprimere in forma chiusa questa funzione, dimostrabile per induzione:
\begin{equation*}
    \lim_{n\to\infty}\left(\displaystyle\frac{n(n+1)}{2}\right)=+\infty
\end{equation*}

Quindi è facile determinare che si tratta di una serie divergente ad infinito positivo. Questo passaggio non d'ora in avanti non sarà più possibile, basterà stabilire il carattere di una serie, senza sapere il valore a cui converga. Solo nel caso della serie telescopica e della serie geometrica sarà possibile determinare il valore della convergenza. 

\subsection{Serie Telescopica}

Il caso più rappresentativo di una serie telescopica è la seguente:
\begin{equation}
    \displaystyle\sum_{k=1}^\infty\frac{1}{k(k+1)}
\end{equation}

Analizzando la somma parziale $s_n$ fino ad un certo valore $n\in\mathbb{N}$ si ha:
\begin{gather*}
    s_n=\displaystyle\sum_{k=1}^n\frac{1}{k(k+1)}=\frac{1}{2}+\cdots+\frac{1}{n(n+1)}
\end{gather*} 

Utilizzando la tecnica dei fratti semplici si può riscrivere la somma parziale come:
\begin{gather*}
    \displaystyle\sum_{k=1}^\infty\frac{1}{k(k+1)}=\sum_{k=1}^\infty\left(\frac{1}{k}-\frac{1}{k+1}\right)
\end{gather*}

In questo modo calcolando la somma parziale si ottiene:
\begin{gather*}
    s_n=\displaystyle\sum_{k=1}^n\left(\frac{1}{k}-\frac{1}{k+1}\right)\\
    \left({1}-\frac{1}{2}\right)+
    \left(\frac{1}{2}-\frac{1}{3}\right)+
    \left(\frac{1}{3}-\frac{1}{4}\right)+
    \cdots+
    \left(\frac{1}{n-1}-\frac{1}{n}\right)+
    \left(\frac{1}{n}-\frac{1}{n+1}\right)
\end{gather*}
Si può notare come l'addendo 1/2 si cancella con l'addendo della differenza successiva, così con 1/3, così per ogni valore della sommatoria, permettendo quindi di scrivere la somma parziale $s_n$ in forma chiusa esprimendo solamente il primo addendo e l'ultimo addendo:
\begin{gather*}
    \left({1}-\cancel{\frac{1}{2}}\right)+
    \left(\cancel{\frac{1}{2}}-\bcancel{\frac{1}{3}}\right)+
    \left(\bcancel{\frac{1}{3}}-\cancel{\frac{1}{4}}\right)+
    \cdots+
    \left(\bcancel{\frac{1}{n-1}}-\cancel{\frac{1}{n}}\right)+
    \left(\cancel{\frac{1}{n}}-\frac{1}{n+1}\right)=
    1-\frac{1}{n+1}
\end{gather*}

Considerando il limite di questa serie si ha:
\begin{equation*}
    \lim_{n\to\infty}\left(1-\displaystyle\frac{1}{n+1}\right)=1
\end{equation*}

In generale una serie telescopica, ha i suoi termini $a_k$ definiti come la differenza tra il termine $k+1$-esimo ed il termine $k$-esimo di una particolare successione $\{A_k\}$:
\begin{equation*}
    \displaystyle\sum_{k=1}^\infty a_k=\sum_{k=1}^\infty\left(A_{k+1}-A_{k}\right)
\end{equation*} 

\subsection{Serie Geometrica}

La serie geometrica è esprimibile come:
\begin{equation*}
    \displaystyle\sum_{k=0}^\infty x^k
\end{equation*}
Come altre serie dipende da un parametro reale $x\in\mathbb{R}$, questo si chiama ragione della serie. 

Si costruisce la somma parziale della serie:
\begin{equation*}
    s_n=\displaystyle\sum_{k=0}^n x^k=1+x+x^2+\cdots+x^n    
\end{equation*}
Si vuole esprimere questa somma in forma chiusa. Per $x=0$, per ogni valore di $n$, la somma vale sempre $1$, poiché:
\begin{gather*}
    \forall n\in\mathbb{N}: s_n=0^0+0^1+0^2+\cdots+0^n=1
\end{gather*}

Un altro caso analogo con $x=1$, si ha la seguente somma parziale:
\begin{gather*}
    \forall n\in\mathbb{N}: s_n=1+1+1+\cdots+1=1\cdot n
\end{gather*}

Questa serie quindi converge per $x=0$, diverge per $x=1$, mentre per altri valori reali non è ancora noto il suo comportamento. 
Si considera ora il caso generale con una ragione diversa da zero e da uno $\forall x\in\mathbb{R}\setminus\{0,1\}$. 
Si considera noto il valore $s_n$:
\begin{equation*}
    s_n=\displaystyle\sum_{k=0}^n x^k=1+x+x^2+\cdots+x^n    
\end{equation*}
Si moltiplica la somma parziale per la ragione e si sottrae alla somma parziale:
\begin{gather*}
    xs_n=x\cdot\displaystyle\sum_{k=0}^n x^k=\sum_{k=1}^nx^k\\
    (1-x)s_n=1+x+x^2+\cdots+x^n-\left(x+x^2+x^3+\cdots+x^n+x^{n+1}\right)=1-x^{n+1}
\end{gather*}

Della prima somma sopravvive solamente il primo addendo $1$, e della seconda l'ultimo addendo $x^{n+1}$, tutti gli altri si cancellano a vicenda, quindi per ottenere il valore della somma parziale si divide il primo ed il secondo membro per il fattore $1-x$:
\begin{equation}
    s_n=\displaystyle\frac{1-x^{n+1}}{1-x}
\end{equation}
Si provvede ora a calcolare il limite di questa somma parziale, dividendolo per la proprietà di linearità, supponendo che questi due limiti risultanti non corrispondono ad una forma indeterminata:
\begin{equation*}
    \lim_{n\to\infty}\displaystyle\frac{1-x^{n+1}}{1-x}=\lim_{n\to\infty}\frac{1}{1-x}-\lim_{n\to\infty}\frac{x^{n+1}}{1-x}
\end{equation*}
Il primo fattore non dipende da $n$, per il limite corrisponde a sé stesso, in seguito si mette in evidenza il fattore $1-x$ ottenendo:
\begin{equation*}
    \displaystyle\frac{1}{1-x}\left(1-\lim_{n\to\infty}x^{n+1}\right)
\end{equation*}
Il comportamento di questo limite dipende dalla ragione:
\begin{gather*}
    \lim_{n\to\infty}x^{n+1}=\begin{cases}
        0 &-1<x<1\\
        +\infty  & x>1\\
        \nexists & x\leq-1
    \end{cases}
\end{gather*}
Per $|x|<1$ la funzione è decrescente, e tende a zero, mentre per $x>1$, la funzione è esponenziale e quindi tende ad infinito. Invece per $x<-1$ si considera il limite, mettendo in evidenza il segno negativo:
\begin{gather*}
    \lim_{n\to\infty}(-|x|)^{n+1}=
    \lim_{n\to\infty}\left[(-1)^{n+1}\cdot |x|^{n+1}\right]
\end{gather*}
Il secondo fattore diverge, mentre il primo fattore oscilla di segno in base alla parità di $n$, per cui questa funzione oscilla di ampiezza sempre maggiore, per cui è irregolare. 
Per $x=-1$ il secondo fattore vale uno, mentre il primo fattore si considera comunque, per cui continua ad oscillare alla stessa ampiezza, quindi rappresenta allo stesso modo un comportamento irregolare. 

Noto il comportamento di questo limite allora il valore della serie geometrica in base alla ragione è dato da:
\begin{equation*}
    \displaystyle\sum_{k=0}^\infty=\begin{cases}
        \displaystyle\frac{\strut 1}{\strut 1-x} &-1<x<1\\
        +\infty & x\geq1\\
        \nexists & x\leq-1
    \end{cases}
\end{equation*}

Questa serie ha un addendo in più poiché parte da $k=0$, per cui esprimendola come una serie partendo da $k=1$:
\begin{equation*}
    \displaystyle\sum_{k=1}^\infty=\displaystyle\sum_{k=0}^\infty-1
\end{equation*}
Il valore di questa successione quindi si comporta analogamente al precedente, considerando l'addendo in più:

\begin{equation*}
    \displaystyle\sum_{k=1}^\infty=\begin{cases}
        \displaystyle\frac{\strut 1}{\strut 1-x}-1 &-1<x<1\\
        +\infty & x\geq1\\
        \nexists & x\leq-1
    \end{cases}
\end{equation*}

\subsection{Criteri di Convergenza}

Esistono vari criteri per determinare la convergenza di una serie numerica. 

%% PRIMO CRITERIO

Il primo criterio definisce una condizione necessaria per la convergenza di una serie geometrica. 
Si considera per ipotesi che una data serie numerica converge, equivale a dire che la serie delle somme parziali converge:
\begin{gather*}
    \displaystyle\sum_{k=1}^\infty a_k\rightarrow s_n=\sum_{k=1}^n a_k\\
    \lim_{n\to\infty}s_n=s
\end{gather*}

Considerando il limite della successione generata aumentando l'indice di uno $n+1$, la combinazione lineare tra queste due successioni converge a zero, poiché rappresentano la stessa successione traslata di un elemento:
\begin{gather*}
    \lim_{n\to\infty}s_{n+1}=s\\
    \lim_{n\to\infty}(s_{n+1}-s_n)=0
\end{gather*}
La successione $s_{n+1}$ corrisponde alla prima successione $s_n$ sommata ad un ulteriore elemento $a_{n+1}$, per cui si può riscrivere la loro differenza come:
\begin{gather*}
    s_{n+1}=s_n+a_{n+1}\\
    s_{n+1}-s_{n}=a_{n+1}
\end{gather*}
Quindi la successione può convergere se il termine infinitesimo della serie converge:
\begin{equation}
    \lim_{n\to\infty}a_{n+1}=0
\end{equation}
Queste rappresenta solo una condizione necessaria, non garantisce che una serie converga, non è quindi una condizione sufficiente. 
Se il termine infinitesimo di una serie non converge, allora la serie non può convergere, quindi diverge oppure è irregolare. 

%% ESEMPIO

Si considera la seguine serie:
\begin{gather*}
    \displaystyle\sum_{k=1}^\infty\frac{3k^2+1}{2k^2+3}
\end{gather*}
Si vuole studiare il carattere della seguente serie utilizzando la condizione appena descritta:
\begin{gather*}
    \lim_{k\to\infty}\displaystyle\frac{3k^2+1}{2k^2+3}=\frac{3}{2}\neq0
\end{gather*}
Questa serie quindi non può converge, potrebbe quindi divergere oppure essere irregolare. 


Esiste un teorema che tratta le successioni numeriche, e si riflette sulle serie. Questo teorema afferma che data una successione monotona crescente $\{a_k\}\uparrow$ o decrescente $\{a_k\}\downarrow$, ovvero se tra il termine $a_k$ ed il termine $a_{k+1}$ esiste una relazione d'ordine, definitivamente, $\forall k\geq k_0$. Allora la successione non può essere irregolare. 

Sia una serie $\sum_{k=0}^\infty a_k$, se $\forall k\geq k_0$ si ha $a_k>0$, allora la serie non è irregolare. Considerando un valore di $n\geq k_0$, si ha:
\begin{gather*}
    s_{n+1}=s_n+\overbrace{a_{n+1}}^{>0}\implies s_{n+1}>s_n
\end{gather*}

Poiché viene sommato ad $s_n$ un valore non nullo positivo, quindi una serie di termini di segno costante positivo corrisponde ad una successione di somme parziali definitivamente monotona crescente. Questa serie allora converge o diverge, è regolare. 

Considerando la serie dell'esempio precedente, con il primo criterio si è confermato che la serie non converge, analizzando i segni della serie si ha:
\begin{gather*}
    3k^2+1>0\,\land\,2k^2+3>0\implies\displaystyle\frac{3k^2+1}{2k^2+3}
\end{gather*}

Quindi la serie è composta da termini definitivamente positivi, quindi la serie non può essere irregolare. Questa serie allora diverge ad infinito positivo. 

%% PROPRIETÀ

Si considerano due serie $\sum a_k$ e $\sum b_k$, convergenti rispettivamente a $\alpha$ e $\beta$, la serie combinazione lineare delle due serie $\sum(Aa_k+Bb_k)$ allora converge al valore $A\alpha+B\beta$. Tramite questa proprietà è possibile scomporre una serie ed analizzare i singoli termini per determinare il carattere della serie da cui derivano. Se una delle due diverge, mentre l'altra converge, allora la serie combinazione lineare diverge, allo stesso segno della serie divergente, per qualunque valore di $A$ e $B$, vale anche se entrambe le serie convergono ad infinito di segno concorde. Se invece entrambe le serie divergono ed il segno discorde, allora questo genera una forma indeterminata e quindi non è possibile determinare il comportamento della serie combinazione lineare. 

\subsubsection{Criterio del Confronto}

Siano $\sum_{k=1}^\infty a_k$ e $\sum_{k=1}^\infty b_k$ due serie di segno definitivamente costante tale che vale la seguente condizione: $\forall k\geq k_0\rightarrow 0<a_k\leq b_k$. Allora se lla serie maggiorante $\sum_{k=1}^\infty b_k$ converge, allora si dimostra che anche la serie minorante $\sum_{k=1}^\infty a_k$ converge. Analogamente vale il duale di questa proprietà se la serie minorante $\sum_{k=1}^\infty a_k$ diverge, allora anche la serie maggiorante $\sum_{k=1}^\infty b_k$ diverge anch'essa. 

%% ESERCIZIO 

Si considera la seguente serie, bisogna analizzarne il comportamento, tramite il criterio del confronto:
\begin{gather*}
    \displaystyle\sum_{k=1}^\infty\frac{|\sin k|}{2^k}
\end{gather*}
Si considera il modulo per avere termini definitivamente positivi, allora il termine generico della serie sarà maggiorato dal termine generico:
\begin{gather*}
    \displaystyle\frac{|\sin k|}{2^k}<\frac{1}{2^k},\,\forall k\in\mathbb{N}
\end{gather*}
È certamente diverso dal termine generico, poiché $k$ non può essere un multiplo di $\pi$. 
La serie maggiorante corrisponde ad una serie geometrica di ragione $1/2$, e converge poiché la ragione è di modulo minore di uno, quindi converge anche la serie minorante. 

Si considera la seguente serie:
\begin{gather*}
    \displaystyle\sum_{k=1}^\infty\frac{2+\cos k}{k}
\end{gather*}
Si può esprimere una serie maggiorante e minorante di questa serie:
\begin{gather*}
    \displaystyle\frac{1}{k}<\frac{2+\cos k}{k}<\frac{3}{k}
\end{gather*}

La serie minorante è una serie armonica, si dimostrerà che diverge a più infinito:
\begin{equation*}
    \displaystyle\sum_{k=1}^\infty\frac{1}{k}=+\infty
\end{equation*}
Mentre la serie armonica generalizzata, per $\alpha\in\mathbb{R}$ diverge solo per certi valori di $\alpha$:
\begin{equation*}
    \displaystyle\sum_{k=1}^\infty\frac{1}{k^\alpha}
\end{equation*}

Si considera la seguente serie:
\begin{gather*}
    \displaystyle\sum_{k=1}^\infty\frac{|\cos k|}{k}
\end{gather*}
La serie maggiorante rappresenta la serie armonica descritta precedentemente, per cui non è possibile utilizzare il criterio del confronto. Non è possibile stabilire il carattere della serie utilizzando questo criterio. 

\subsubsection{Criterio della Radice}

Sia una serie definitivamente monotona crescente $\sum_{k=1}^\infty a_k,\,\forall k\geq k_0$. 
Si basa sulla definizione di una successione ausiliaria. Per costruire questa successione ausiliaria si considera, da $k>k_0$, $\{\sqrt[k]{a_k}\}$. Si considera per ipotesi che questa successione sia regolare:
\begin{equation*}
    \lim_{k\to\infty}\sqrt[k]{a_k}=\Lambda
\end{equation*}

Se questo limite converge ad un valore positivo maggiore di uno, oppure diverge ad infinito positivo, allora è dimostrabile che la serie di partenza diverge. 
Se questo limite è positivo e minore di uno, allora la serie converge. Mentre se il limite è esattamente pari ad uno, allora il problema rimane aperto, e non è possibile stabilire il comportamento della serie tramite questa successione. 

\begin{equation}
    \displaystyle\sum_{k=1}^\infty a_k:\begin{cases}
        \text{diverge}&\Lambda > 1\,\lor\Lambda=+\infty\\
        \text{converge}&0\leq\Lambda<1\\
        ?&\Lambda=1
    \end{cases}
\end{equation}

Considerando la seguente serie, determinare il carattere tramite il metodo della radice:
\begin{gather*}
    \displaystyle\sum_{k=1}^\infty\left(\frac{1}{k}\right)^k
\end{gather*}
La sua serie ausiliaria è:
\begin{gather*}
    \displaystyle\left\{\sqrt[k]{\left(\frac{1}{k}\right)^{k}}\right\}=\left\{\frac{1}{k}\right\}
\end{gather*}
Questa successione corrisponde alla serie armonica, e diverge quindi la serie di partenza diverge. 

\subsubsection{Criterio del Rapporto}

Sia una serie definitivamente monotona crescente $\sum_{k=1}^\infty a_k,\,\forall k\geq k_0$. 
Si basa sulla definizione di una successione ausiliaria. Per costruire questa successione ausiliaria si considera, da $k>k_0$:
\begin{equation*}
    \displaystyle\left\{\frac{a_{k+1}}{a_k}\right\}
\end{equation*}
Si considera per ipotesi che questa successione sia regolare:
\begin{equation*}
    \lim_{k\to\infty}\sqrt[k]{a_k}=\Lambda
\end{equation*}

Esistono casi dove la successione ausiliaria creata è irregolare, e quindi non è possibile utilizzare questi criteri. %% !! base55555555555555555555555555
Analogamente al criterio precedente, in base al valore del limite la serie diverge, converge, oppure non è possibile determinarne il carattere:
\begin{equation}
    \displaystyle\sum_{k=1}^\infty a_k:\begin{cases}
        \text{diverge}&\Lambda > 1\,\lor\Lambda=+\infty\\
        \text{converge}&0\leq\Lambda<1\\
        ?&\Lambda=1
    \end{cases}
\end{equation}

%% ESEMPIO

Si considera la seguente serie:
\begin{gather*}
    \displaystyle\sum_{k=1}^\infty\frac{2^k}{k!}
\end{gather*}

La sua serie ausiliaria è quindi:
\begin{gather*}
    \displaystyle\frac{\frac{2^{k+1}}{(k+1)!}}{\frac{2^k}{k!}}=
    \frac{2^{k+1}}{(k+1)!}\cdot\frac{k!}{2^k}=\cancelto{\frac{1}{k+1}}{\frac{k!}{(k+1)!}}\cdot\cancelto{2}{\frac{2^{k+1}}{2^k}}=\frac{2}{k+1}\\
    \lim_{k\to\infty}\frac{2}{k+1}=0
\end{gather*}

La serie quindi converge, invece se i termini della serie fossero inversi:
\begin{gather*}
    \displaystyle\sum_{k=1}^\infty\frac{k!}{2^k}
\end{gather*}

\subsubsection{Criterio dell'Integrale}
Sia una serie definitivamente monotona crescente $\sum_{k=1}^\infty a_k,\,\forall k\geq k_0$. 
Si considera la funzione associata che sostituisce $\{a_k\}\rightarrow\{f(x)\}$. La funzione associata deve essere decrescente a partire da $k_0$: $f(x)\downarrow \forall x>k_0$. Sia:
\begin{equation*}
    t_n=\displaystyle\int_{k_0}^n f(x)\mathrm{d}x
\end{equation*}

Allora la successione definita su $t_n$: $\{t_n\}$ e la serie $\sum a_k$ hanno lo stesso carattere. 

Si applica il criterio dell'integrale sulla serie armonica generalizzata, questa soddisfa l'ipotesi da $k_0=1$:
\begin{gather*}
    \displaystyle\sum_{k=1}^\infty\frac{1}{k^\alpha},\,\alpha\in\mathbb{R}\\
    f(x)=\displaystyle\frac{1}{x^\alpha}
\end{gather*}
Se $\alpha$ fosse negativo, la funzione associata sarebbe monotona crescente, ma se fosse negativo la serie sarebbe $\sum k^{|\alpha|}$, per cui diverge. 
Per $\alpha=0$, tutti i termini della serie sono pari ad uno, quindi il termine generico è uno e quindi la serie diverge allo stesso modo. Quindi solo per $\alpha>0$ la serie converge, e quindi solo per questo caso ha senso lo studio della serie armonica generalizzata

Nel caso $\alpha>0$ la condizione necessaria è soddisfatta, quindi la funzione potrebbe convergere. Si studia il suo carattere con il criterio dell'integrale, si ha la seguente serie ausiliaria:
\begin{gather*}
    t_n=\displaystyle\int_{1}^n\frac{1}{x^\alpha}\mathrm{d}x=\begin{cases}
        \bigg|\log(x)\bigg|_1^n &\alpha=1\\
        \displaystyle\frac{1}{1-\alpha}\bigg|x^{1-\alpha}\bigg|_1^n&\alpha\neq1
    \end{cases}=\begin{cases}
        \log(n)&\alpha=1\\
        \displaystyle\frac{1}{1-\alpha}\left(n^{1-\alpha}-1\right)&\alpha\neq1
    \end{cases}
\end{gather*}

Per $\alpha=1$, la successione diverge, poiché il logaritmo è una funzione monotona crescente, mentre per $\alpha\neq1$:
\begin{gather*}
    \displaystyle\frac{1}{1-\alpha}\lim_{n\to\infty}\left(n^{1-\alpha}-1\right)=\begin{cases}
        +\infty&\alpha<1\\
        \displaystyle\frac{1}{1-\alpha}&\alpha>1
    \end{cases}
\end{gather*}
La serie quindi diverge per $\alpha<1$, mentre converge per $\alpha>1$

Il carattere della serie armonica generalizzata è quindi dato dal seguente:
\begin{equation}
    \displaystyle\sum_{k=1}^\infty\frac{1}{k^\alpha}:\begin{cases}
        \text{diverge}&\alpha\le1\\
        \text{converge}&\alpha>1        
    \end{cases}
\end{equation}


Conviene applicare il criterio dell'integrale quando la funzione associata è facilmente integrabile. Considerata la seguente serie:
\begin{equation*}
    \displaystyle\sum_{k=1}^{\infty}\frac{\text{arctan}\, k}{1+k^2}
\end{equation*}

Per valutare il carattere di questa serie si potrebbe considerare il criterio del confronto, poiché l'arcotangente è una funzione superiormente limitata da $\pi/2$:
\begin{gather*}
    \displaystyle\frac{\text{arctan}\,k}{1+k^2}<\frac{\pi/2}{1+k^2}<\frac{\pi}{2k^2}
\end{gather*}

Considerando la serie maggiorante, questa è una serie armonica generalizzata con parametro $\alpha=2>1$, questa converge, quindi per il criterio del confronto converge anche la serie di partenza. 

In questo caso funziona anche il criterio dell'integrale. La sua funzione associata è:
\begin{gather*}
    f(x)=\displaystyle\frac{\text{arctan}\, x}{1+x^2}
\end{gather*}

Si può notare come la funzione associata è il prodotto dell'arcotangente per la sua derivata, quindi l'integrale è immediato. Per utilizzare il criterio dell'integrale bisogna determinare che la funzione è monotona decrescente, si analizza quindi la sua derivata:
\begin{gather*}
    \displaystyle\frac{\mathrm{d}f(x)}{\mathrm{d}x}=\displaystyle\frac{\cancelto{1}{\frac{1}{1+x^2}(1+x^2)}-2x\text{arctan}\,x}{(1+x^2)^2}\\
    1-2x\text{arctan}\,x <0
\end{gather*}
Il numeratore è sempre negativo per $x\in(1,\infty)$, per cui la funzione ausiliaria è monotona decrescente ed è possibile utilizzare il criterio dell'integrale:
\begin{gather*}
    t_k=\displaystyle\int_{1}^kf(x)\mathrm{d}x=\int_1^k\frac{\text{arctan}\,x}{1+x^2}\mathrm{d}x\\
    u=\text{arctan}\,x\rightarrow\int_1^{\text{arctan}\,k}u \mathrm{d}u=\bigg|\frac{u^2}{2}+c\bigg|_1^{\text{arctan}\,k}= %% TODO finire
\end{gather*}

\subsubsection{Criterio di Condensazione}

Sia una serie di segno definitivamente positivo: $\sum_{k=1}^\infty a_k | a_k>0\,\forall k\geq k_0$. Si considera la serie monotona decrescente $a_k\downarrow$, allora la serie $\sum 2^ka_{2^k}$ ha lo stesso carattere della serie di partenza. Generalmente si applica solo nel caso dove in $a_k$ è presente un fattore logaritmo di $k$. 

Si considera la serie:
\begin{equation*}
    \displaystyle\sum_{k=1}^\infty\frac{\ln k}{k}
\end{equation*}

Si calcola la derivata della sua funzione ausiliaria:
\begin{gather*}
    \displaystyle f(x)=\frac{\ln x}{x}\\
    f'(x)=\displaystyle\frac{1-\ln x}{x^2}<0
\end{gather*}
Si è verificata l'ipotesi della monotonia della serie. 
Si applica quindi il criterio di condensazione:
\begin{gather*}
    \displaystyle\sum_{k=1}^\infty \cancel{2^k}\frac{\ln 2^k}{\cancel{2^k}}
\end{gather*}

Sfruttando le proprietà del logaritmo questo diventa:
\begin{gather*}
    \displaystyle\sum_{k=1}^\infty k\cdot{\ln 2}=\ln 2\displaystyle\sum_{k=1}^\infty k
\end{gather*}

Non verificando l'ipotesi di convergenza, non può convergere. 

Si considera ora la serie:
\begin{gather*}
    \displaystyle\sum_{k=1}^\infty\frac{\ln k}{k^2}\\
    \displaystyle f(x)=\frac{\ln x}{x^2}\\
    f'(x)=\displaystyle\frac{x-2x\ln x}{x^4}=\frac{\cancel{x}(1-2\ln x)}{x^{\cancelto{3}{4}}}<0
\end{gather*}

Si applica il criterio di condensazione:
\begin{gather*}
    \displaystyle\sum_{k=1}^\infty \cancel{2^k}\frac{\ln 2^k}{(2^k)^{\cancel{2}}}=\ln2\sum_{k=1}^\infty\frac{k}{2^k}
\end{gather*}
Si studia il criterio del rapporto per poter determinarne il carattere:
\begin{gather*}
    \displaystyle\frac{\frac{k+1}{2^{k+1}}}{\frac{k}{2^k}}=\frac{k+1}{k}\frac{\cancel{2^k}}{2^{\cancel{k+1}}}=\frac{1}{2}\frac{k+1}{k}\\
    \frac{1}{2}\lim_{k\to\infty}\frac{k+1}{k}=\frac{1}{2}<1
\end{gather*}
Poiché $a_{k+1}/a_{k}<1$ la serie converge, e quindi per il criterio di condensazione anche la serie di partenza converge.  

\subsubsection{Criterio del Confronto Asintotico}

In questo criterio si considerano due serie, entrambe a termini definitivamente positivi $\sum_{k=1}^\infty a_k \land \sum_{k=1}^\infty b_k | a_k>0,\,b>0\,\forall k\geq k_0$. Allora se il limite del rapporto all'infinito converge ad un numero reale non nullo, allora hanno lo stesso carattere
\begin{gather*}
    \lim_{k\to\infty}\frac{a_k}{b_k}=l\in\mathbb{R}\setminus\{0\}
\end{gather*}
Se $l=1$ allora le due successioni, analogamente per le funzioni, si dicono asintoticamente equivalenti per $k\to\infty$: $a_k$~$b_k$. %% TODO tilde c'è?
Spesso si userà questo simbolo generalizzandolo come due funzioni asintoticamente equivalenti a meno di una costante, poiché è di interessa solamente il carattere e non il valore effettivo esplicito di una serie. 
Se invece il rapporto converge a zero e la serie $\sum b_k$ converge, allora anche la serie $\sum a_k$ converge, ovvero $a_k$ è un infinitesimo di ordine superiore a $b_k$. 
Se invece il limite di questo rapporto tende ad infinito, e la serie $\sum b_k$ diverge, allora divergerà anche la serie $\sum a_k$, $b_k$ è un infinitesimo di ordine superiore di $a_k$. 

Si considera la seguente serie:
\begin{gather*}
    \displaystyle\sum_{k=1}^\infty\sin\left(\frac{1}{k}\right)
\end{gather*}
È soddisfatta la condizione necessaria per la convergenza, avendo:
\begin{gather*}
    \displaystyle\lim_{k\to\infty}\sin\left(\frac{1}{k}\right)=0
\end{gather*}

Bisogna ora trovare una serie $\sum b_k$, tale che:
\begin{gather*}
    \displaystyle\lim_{k\to\infty}\frac{\sin\left(\frac{1}{k}\right)}{b_k}=l\in\mathbb{R}\setminus\{0\}
\end{gather*}
Per il limite notevole $\sin x/x$, questa serie è $b_k=1/k$, per cui il limite converge ad uno:
\begin{gather*}
    \displaystyle\lim_{k\to\infty}\frac{\sin\left(\frac{1}{k}\right)}{\frac{1}{k}}=1
\end{gather*}

Queste due serie sono asintoticamente equivalenti tra di loro:
\begin{gather*}
    \displaystyle\sum_{k=1}^\infty\sin\left(\frac{1}{k}\right)\sim\sum_{k=1}^\infty\frac{1}{k}
\end{gather*}
La serie di destra è una serie armonica e diverge, quindi diverge anche la serie di partenza. 5

Data ora la seguente serie:
\begin{gather*}
    \displaystyle\sum_{k=1}^\infty\sin\left(\frac{1}{k^2}\right)
\end{gather*}

I termini sono sempre positivi, ed è soddisfatta la condizione necessaria, la serie asintoticamente equivalente è data da:
\begin{gather*}
    \displaystyle\sum_{k=1}^\infty\sin\left(\frac{1}{k^2}\right)\sim\displaystyle\sum_{k=1}^\infty\frac{1}{k^2}
\end{gather*}

Questa è una serie armonia generalizzata per $\alpha=2>1$, quindi converge. Allora anche la serie di partenza converge.  


Si considera la seguente serie:
\begin{gather*}
    \displaystyle\sum_{k=1}^\infty\ln\left(\frac{k+1}{k}\right)
\end{gather*}

La serie ha termini di segno definitivamente positivo, poiché l'argomento è certamente maggiore di uno: $k+1>k$. Si scrive in forma equivalente l'argomento per evidenziare il limite notevole:
\begin{gather*}
    \displaystyle\sum_{k=1}^\infty\ln\left(\frac{k+1}{k}\right)=
    \displaystyle\sum_{k=1}^\infty\ln\left(1+\frac{1}{k}\right)\sim\sum_{k=1}^\infty\frac{1}{k}
\end{gather*}

Questa serie ha lo stesso carattere di una serie armonica divergente, quindi diverge. 

Si considera la seguente serie:
\begin{gather*}
    \displaystyle\sum_{k=1}^\infty\left(e^{1/k}-1\right)\sim\sum_{k=1}^\infty\frac{1}{k}
\end{gather*}
Per il limite notevole è asintoticamente equivalente ad una serie armonica divergente, quindi diverge. 

Considerando ora la seguente serie:
\begin{gather*}
    \displaystyle\sum_{k=1}^\infty\left(e^{1/\sqrt{k}}-1\right)\sim\sum_{k=1}^\infty\frac{1}{k^{1/2}}
\end{gather*}
Essendo asintoticamente equivalente ad una serie armonica generalizzata divergente, diverge. 
Per avere avere una serie convergente, deve avere come argomento almeno un $k^\alpha$, con $\alpha>1$.. 


Si considera la seguente serie:
\begin{gather*}
    \displaystyle\sum_{k=1}^\infty\ln\left(\frac{k^2+5k-2}{k^2+3}\right)
\end{gather*}
Sicuramente è una serie a termini positivi, si vuole quindi riscrivere l'argomento del logaritmo nella forma $1+x$:
% !! 55555555555555555555
% ?? 55555555555555555555555555555555555555555555558555555555555555555555555555555555555555555555555555555555555
% todo 55555555555555555555255555555555555555525555555585555555555555555554555555555555555555555565565555565656565555555555555555555555555555
\begin{gather*}
    \displaystyle\frac{k^2+5k-2}{k^2+3}=1+\frac{5k-5}{k^2+3}
\end{gather*}
Questa serie si comporta in modo equivalente alla serie:
\begin{gather*}
    \displaystyle\sum_{k=1}^\infty\ln\left(\frac{k^2+5k-2}{k^2+3}\right)=
    \displaystyle\sum_{k=1}^\infty\ln\left(1+\frac{5k-5}{k^2+3}\right)\sim\sum_{k=1}^\infty\frac{5k-5}{k^2+3}\sim\sum_{k=1}^\infty\frac{5k}{k^2}=5\sum_{k=1}^\infty\frac{\cancel{k}}{k^{\cancel{2}}}
\end{gather*}
Essendo asintoticamente equivalente ad una serie armonica divergente, anche essa diverge. 

\subsubsection{Criterio di Leibniz}

Una prima famiglia di serie a termini di segno alterni permette comunque di poter calcolare. Considerata una serie con $a_k>0\forall k>k_0$, si esprime il fattore che rende la serie alterna:
\begin{gather*}
    \displaystyle\sum_{k=1}^\infty(-1)^ka_k
\end{gather*}

Il criterio di Leibniz suppone di avere una serie del genere, siano i termini della serie decrescenti, tendenti a 0: $\{a_k\}\downarrow0$, allora la serie converge. 


Detta $S$ la somma della serie:
\begin{gather*}
    s=\displaystyle\sum_{k=0}^\infty(-1)^ka_k
\end{gather*}

All'aumentare di $n$, la somma $s_n$ dei primi $n$ elementi della serie, la serie oscilla sorpassando in positivo o in negativo la somma $s$, di valori sempre più piccoli, convergendo al valore $s$:

%% TODO asse reale con S1, S2, ... S oscillazione
\begin{gather*}
    |s-s_1| < |s_1-s_2|\\
    |s-s_n| \leq |s_n-s_{n-1}|=|a_n|
\end{gather*}

La somma $s_n$ si posizione al lato opposto di $s_{n-1}$ e si trova ad una posizione più vicina alla somma $s$ rispetto al precedente, quindi ha una distanza minore. 

\subsubsection{Criterio di Convergenza Assoluta}

Il criterio di Leibniz si può utilizzare solo quando le serie hanno segno alterno, mentre, per altri tipo di serie in cui il segno non è né definitivamente positivo, o negativo, e né alterno, è necessario un altro criterio per poter determinarne il carattere. 

Un criterio generale consiste nell'analizzare il carattere della serie $|a_k|$, questa serie risultante è a termini di segno costante positivo:
\begin{gather*}
    \displaystyle\sum_{k=1}^\infty|a_k|
\end{gather*}

Un teorema dimostrabile afferma che se una serie ha termini di segno non costante $\sum a_k$, se la serie formata mettendo in valore assoluto i termini della prima $\sum|a_k|$ converge, allora converge anche la serie iniziale. Si dice che converge in modo assoluto, o assolutamente. 

Se la serie $\sum a_k$ converge assolutamente, allora la serie converge semplicemente, la convergenza studiata con i criteri precedenti. 

\subsection{Serie di Potenze}

In generale una serie di potenza viene espressa, partendo da 0 o 1, come:
\begin{gather*}
    \displaystyle\sum_{k=0}^\infty a_k(x-x_0)^k
\end{gather*}
Con $x_0\in\mathbb{R}$, si dice serie di potenze con centro in $x_0$. Le serie di potenze sono molto importanti e di facile analisi, anche se il calcolo del loro limite non è triviale. 

Quando si analizzano serie di potenze si parla di intervallo di convergenza, poiché si analizza per quali valori del parametro $x$ la serie converge. Una serie geometrica è una particolare serie di potenze centrata in $x_0=0$, con $a_k=1$ per ogni valore di $k$, di cui si conosce anche la somma, per valori del parametro $x\in(-1,1)$:
\begin{gather*}
    \displaystyle\sum_{k=1}^\infty x^k=\frac{1}{1-x}
\end{gather*}

Tutte le volte che si ha una serie di potenze, la prima cosa che bisogna calcolare è il cosiddetto raggio di convergenza. La sua definizione è molto più ampia, ma studiando casi particolari non sarà necessario approfondirlo. Per calcolarlo si introduce una successione ausiliaria:
\begin{gather*}
    \left\{\displaystyle\left|\frac{a_k}{a_{k+1}}\right|\right\}
\end{gather*}
Questa serie si riferisce al criterio del rapporto, alternativamente si può usare la serie ausiliaria: %% TODO quale criterio
\begin{gather*}
    \left\{\displaystyle\frac{1}{\sqrt[k]{a_k}}\right\}
\end{gather*}
Bisogna dare per ipotesi che queste successioni siano regolari, per $k\to\infty$. Questo valore $R$ si trova calcolando il limite per $k\to\infty$ di una di queste due successioni:
\begin{equation}
    R=\lim_{k\to\infty}\displaystyle\left|\frac{a_k}{a_{k+1}}\right|=\lim_{k\to\infty}\frac{1}{\sqrt[k]{a_k}}=\begin{cases}
        +\infty\\
        0\\
        \in\mathbb{R}^+
    \end{cases}
\end{equation}

Se il limite esiste assume uno di questi tre possibili valori. Una volta determinato il raggio di convergenza, esiste un teorema che stabilisce dove questa serie risulta sicuramente convergente. 

Sia $R$ il raggio di convergenza della serie di potenze $a_k$ con centro in $x_0$: $\sum a_k(x-x_0)^k$. Allora si può affermare:
\begin{enumerate}
    \item Se $R=0$, la serie converge solo nel centro $x_0$. 
    \item Se $R=+\infty$, la serie converge in $\mathbb{R}$, ovvero qualunque sia il valore reale del parametro $x$. 
    \item Se $R\in\mathbb{R}^+$, finito positivo e diverso da zero, la serie converge assolutamente, e quindi semplicemente, in $(x_0-R, x_0+R)$. Inoltre la serie non converge nell'intervallo $(-\infty,x_0-R)$ e $(x_0+R, +\infty)$. 
\end{enumerate}

Nell'ultimo caso $R$ rappresenta il centro dell'intervallo di convergenza, se il raggio è zero, contiene solo il centro, se è infinito, contiene tutti i numeri reali. 
Per determinare il valore al limite dell'intervallo $x=x_0\pm R$, bisogna sostituire questo valore ad $x$ e studiare per via diretta il carattere della serie risultante. Per questo si distingue tra intervallo di convergenza ed insieme di convergenza, poiché potrebbero non coincidere, potrebbero comprendere gli estremi. Coincidono solo se ai valori estremi la serie non converge. 


\subsection*{Esercizi}

Studiare il carattere della serie numerica:
\begin{gather*}
    \displaystyle\sum_{k=1}^\infty\frac{k^2+1}{k^3+1}\ln\left(1+\frac{1}{k}\right)
\end{gather*}
Per il limite notevole, il carattere di questa serie è asintoticamente equivalente alla serie:
\begin{gather*}
    \displaystyle\sum_{k=1}^\infty\frac{k^2+1}{k^3+1}\ln\left(1+\frac{1}{k}\right)\sim
    \displaystyle\sum_{k=1}^\infty\frac{k^2+1}{k^3+1}\frac{1}{k}=
    \displaystyle\sum_{k=1}^\infty\frac{k^2+1}{k^4+k}
\end{gather*}
Il numeratore si comporta come $k^2$, sono asintoticamente equivalenti, mentre il denominatore è asintoticamente equivalente a $k^4$, quindi la serie è equivalente a:
\begin{gather*}
    \displaystyle\sum_{k=1}^\infty\frac{k^2+1}{k^4+k}\sim
    \displaystyle\sum_{k=1}^\infty\frac{\cancel{k^2}}{k^{\cancelto{2}{4}}}
\end{gather*}

Questa è una serie armonica generalizzata con $\alpha=2>1$, per cui converge. 


Si può realizzare una serie di stesso carattere sostituendo il fattore trascendente con un altro sempre appartenente ad un limite notevole:
\begin{gather*}
    \displaystyle\sum_{k=1}^\infty\frac{k^2+1}{k^3+1}\left(e^{1/k}-1\right)\sim
    \displaystyle\sum_{k=1}^\infty\frac{k^2+1}{k^3+1}\frac{1}{k}=
    \displaystyle\sum_{k=1}^\infty\frac{k^2+1}{k^4+k}\sim
    \displaystyle\sum_{k=1}^\infty\frac{\cancel{k^2}}{k^{\cancelto{2}{4}}}
\end{gather*}



Studiare il carattere della seguente serie:
\begin{gather*}
    \displaystyle\sum_{k=1}^\infty\frac{k^4+k^3}{k^5+1}\ln\left(\frac{k+1}{k}\right)=
    \displaystyle\sum_{k=1}^\infty\frac{k^4+k^3}{k^5+1}\ln\left(1+\frac{1}{k}\right)\sim
    \displaystyle\sum_{k=1}^\infty\frac{k^4+k^3}{k^5+1}\frac{1}{k}=
    \displaystyle\sum_{k=1}^\infty\frac{k^5+k^4}{k^6+k}\sim
    \displaystyle\sum_{k=1}^\infty\frac{\cancel{k^5}}{k^{\cancel{6}}}=
    \displaystyle\sum_{k=1}^\infty\frac{1}{k}
\end{gather*}
Essendo asintoticamente equivalente ad una serie armonica divergente, la serie diverge. 


Studiare il carattere della seguente serie:
\begin{gather*}
    \displaystyle\sum_{k=1}^\infty\frac{5^k}{k^2(2^k+4^k)}
\end{gather*}
Poiché il calcolo della condizione necessaria di convergenza è complessa, per risolvere l'esercizio si suppone sia corretta e si utilizza subito uno dei criteri:
\begin{gather*}
    \displaystyle\sum_{k=1}^\infty\frac{5^k}{k^2(2^k+4^k)}=
    \displaystyle\sum_{k=1}^\infty\frac{5^k}{k^24^k(1/2^k+1)}
\end{gather*}
Si mette in evidenza l'esponenziale con base maggiore, poiché il fattore con base minore di uno per $k\to\infty$ tende a zero, quindi questa serie è asintoticamente equivalente a:
\begin{gather*}
    \displaystyle\sum_{k=1}^\infty\frac{5^k}{k^24^k(1/2^k+1)}\sim
    \displaystyle\sum_{k=1}^\infty\frac{1}{k^2}\left(\frac{5}{4}\right)^k
\end{gather*}
Si utilizza ora il criterio del rapporto, si determina la serie ausiliaria:
\begin{gather*}
    \displaystyle\frac{\frac{1}{(k+1)^2}\left(\frac{5}{4}\right)^{\cancel{k+1}}}{\frac{1}{k^2}\cancel{\left(\frac{5}{4}\right)^k}}=\frac{5}{4}\left(\frac{k}{k+1}\right)^2\\
    \frac{5}{4}\lim_{k\to\infty}\left(\frac{k}{k+1}\right)^2=\frac{5}{4}>1
\end{gather*}
Quindi la serie diverge.  


Studiare il carattere della seguente serie:
\begin{gather*}
    \displaystyle\sum_{k=1}^\infty\frac{k^23^k}{2^k+4^k}
\end{gather*}
Questo esercizio è molto simile al precedente, ed analogamente si manipola il denominatore per rimuovere la somma di esponenziali:
\begin{gather*}
    \displaystyle\sum_{k=1}^\infty\frac{k^23^k}{2^k+4^k}=
    \displaystyle\sum_{k=1}^\infty\frac{k^23^k}{4^k(1/2^k+1)}\sim
    \displaystyle\sum_{k=1}^\infty\frac{k^23^k}{4^k}=
    \displaystyle\sum_{k=1}^\infty k^2\left(\frac{3}{4}\right)^k
\end{gather*}
Si utilizza analogamente alla precedente il criterio del rapporto; la sua serie ausiliaria è:
\begin{gather*}
    \displaystyle\frac{(k+1)^2\left(\frac{3}{4}\right)^{\cancel{k+1}}}{k^2\cancel{\left(\frac{3}{4}\right)^k}}=\frac{3}{4}\left(\frac{k+1}{k}\right)^2\\
    \lim_{k\to\infty}\frac{3}{4}\left(\frac{k+1}{k}\right)^2=\frac{3}{4}<1
\end{gather*}
Poiché il rapporto tendente all'infinito è minore di uno, allora la serie converge. 


Studiare il carattere della seguente serie:
\begin{gather*}
    \displaystyle\sum_{k=1}^\infty\frac{k^2+1}{k^3+1}\ln\left(\frac{k+1}{k}\right)
\end{gather*}
Questo è esattamente uguale al primo esercizio, quindi anch'esso diverge, l'argomento del logaritmo è scritto in una forma equivalente. 


Studiare il carattere della seguente serie:
\begin{gather*}
    \displaystyle\sum_{k=1}^\infty\frac{2^k(k^2+\sin e^k)}{3^k}\sim
    \displaystyle\sum_{k=1}^\infty\frac{2^kk^2}{3^k}=
    \displaystyle\sum_{k=1}^\infty k^2\left(\frac{2}{3}\right)
\end{gather*}
Nel numeratore si ha $k^2+\sin e^k$~$k^2$, poiché l'oscillazione di $\sin e^k$ viene smorzata all'infinito da $k^2$, può essere dimostrato tramite il criterio del confronto. 
Si applica il criterio del rapporto:
\begin{gather*}
    \displaystyle\frac{(k+1)^2\left(\frac{2}{3}\right)^{\cancel{k+1}}}{k^2\cancel{\left(\frac{2}{3}\right)^k}}=\frac{3}{4}\left(\frac{k+1}{k}\right)^2\\
    \lim_{k\to\infty}\frac{2}{3}\left(\frac{k+1}{k}\right)^2=\frac{2}{3}<1
\end{gather*}
Essendo il limite minore di uno, la serie converge. 

%% TODO SEGNARE EQUAZIONE

La formula di Stirling afferma che un fattoriale di un numero intero si comporta al limite per $k\to\infty$ come:
\begin{gather*}
    \displaystyle\lim_{k\to\infty}\frac{k!}{\sqrt{2k\pi}\left(\frac{k}{e}\right)^k}=1\\
    k!\sim\sqrt{2k\pi}\left(\frac{k}{e}\right)^k\tageq
\end{gather*}



Studiare il carattere della seguente serie:
\begin{gather*}
    \displaystyle\sum_{k=1}^\infty\frac{\sin k^3-k^{3/5}}{k^{1/4}\ln(k^k+k!)}
\end{gather*}
Il denominatore è sempre positivo, mentre nel numeratore il seno può essere al massimo pari ad uno, mentre l'altro fattore $k^{3/5}$ è sicuramente maggiore di uno da un certo $k_0$, quindi si può esprimere come:
\begin{gather*}
    -\displaystyle\sum_{k=1}^\infty\frac{k^{3/5}-\sin k^3}{k^{1/4}\ln(k^k+k!)}
\end{gather*}

Il numeratore può essere sostituito con $k^{3/5}$, analogamente alla precedente utilizzando il criterio del confronto può essere dimostrato come il seno diviso questo fattore tende a zero per $k\to\infty$, quindi la serie si può riscrivere come:
\begin{gather*}
    -\displaystyle\sum_{k=1}^\infty\frac{k^{3/5}-\sin k^3}{k^{1/4}\ln(k^k+k!)}\sim
    -\displaystyle\sum_{k=1}^\infty\frac{k^{3/5}}{k^{1/4}\ln(k^k+k!)}=
    -\displaystyle\sum_{k=1}^\infty \frac{k^{7/20}}{\ln(k^k+k!)}
\end{gather*}
Essendo $k^k$ un infinito di ordine superiore a $k!$, si può esprimere il denominatore in modo asintoticamente equivalente come:
\begin{gather*}
    -\displaystyle\sum_{k=1}^\infty \frac{k^{7/20}}{\ln(k^k+k!)}=
    -\displaystyle\sum_{k=1}^\infty \frac{k^{7/20}}{\ln\left[k^k\left(1+\frac{k!}{k^k}\right)\right]}=
    -\displaystyle\sum_{k=1}^\infty \frac{k^{7/20}}{\ln k^k+\ln\left(1+\frac{k!}{k^k}\right)}\sim
    -\displaystyle\sum_{k=1}^\infty \frac{k^{7/20}}{\ln k^k}    
\end{gather*}
Si dimostra:
\begin{gather*}
    \lim_{k\to\infty}\frac{\ln k^k+\ln\left(1+\frac{k!}{k^k}\right)}{\ln k^k}=
    \lim_{k\to\infty}\left[\cancelto{1}{\frac{\ln k^k}{\ln k^k}}+\frac{\ln\left(1+\frac{k!}{k^k}\right)}{\ln k^k}\right]\\
    \lim_{k\to\infty}\ln\left(1+\frac{k!}{k^k}\right)=0\rightarrow
    \lim_{k\to\infty}\frac{\ln k^k+\ln\left(1+\frac{k!}{k^k}\right)}{\ln k^k}=1
\end{gather*}
Si può riscrivere come:
\begin{gather*}
    -\displaystyle\sum_{k=1}^\infty\frac{k^{7/20}}{k}\frac{1}{\ln k}=
    -\displaystyle\sum_{k=1}^\infty\frac{1}{k^{13/20}\ln k}
\end{gather*}

In questa situazione si applica il criterio di condensazione creando la serie ausiliaria:
\begin{gather*}
    -\displaystyle\sum_{k=1}^\infty2^k\frac{1}{(2^k)^{13/20}\ln 2^k}=
    -\displaystyle\sum_{k=1}^\infty\frac{2^k}{(2^k)^{13/20}}\frac{1}{k\ln 2}=
    -\displaystyle\sum_{k=1}^\infty\frac{2^{7k/20}}{k\ln 2}=-\frac{1}{\ln 2}\sum_{k=1}^\infty\frac{2^{7k/20}}{k}
\end{gather*}
Applicando il criterio del rapporto si ha:
\begin{gather*}
    \displaystyle\frac{2^{\cancel{7(k+1/20)}}}{k+1}\frac{k}{\cancel{2^{7k/20}}}=2^{7/20}\frac{k}{k+1}\\
    \lim_{k\to\infty}2^{7/20}\frac{k}{k+1}=2^{7/20}<1
\end{gather*}
Poiché il limite tende ad un valore minore di uno, la serie converge. 



Studiare il carattere della serie:
\begin{gather*}
    \displaystyle\sum_{k=0}^\infty\left(\sqrt[3]{k^3+1}-k\right)
\end{gather*}

Si considera il prodotto notevole della differenza di due cubi, per determinare se questa serie soddisfa almeno la condizione necessaria per la convergenza:
\begin{gather*}
    \displaystyle a=\sqrt[3]{k^3+1}\land b=k\\
    \displaystyle a^3-a^3=(a-b)(a^2+ab+B^b)\\
    \displaystyle a-b=\frac{a^3-b^3}{a^2+ab+b^2}\\
    \displaystyle\sqrt[3]{k^3+1}-k=\frac{\cancel{k^3}+1-\cancel{k^3}}{\left(\sqrt[3]{k^3+1}\right)^2+k\sqrt[3]{k^3+1}+k^2}\\
    \displaystyle\sum_{k=0}^\infty\left(\sqrt[3]{k^3+1}-k\right)=
    \sum_{k=0}^\infty\frac{1}{\left(\sqrt[3]{k^3+1}\right)^2+k\sqrt[3]{k^3+1}+k^2}
\end{gather*}
Ragruppando gli addendi di potenza più grande:
\begin{gather*}
    \sum_{k=0}^\infty\frac{1}{\left(\sqrt[3]{k^3+1}\right)^2+k\sqrt[3]{k^3+1}+k^2}
    \displaystyle\sum_{k=0}^\infty\frac{1}{\sqrt[3]{k^6\left(1+\frac{2}{k^3}+\frac{1}{k^6}\right)}+k\sqrt[3]{k^3\left(1+\frac{1}{k^3}\right)}+k^2}\\
    \sum_{k=0}^\infty\frac{1}{k^2\left(\sqrt[3]{1+\frac{2}{k^3}+\frac{1}{k^6}}+\sqrt[3]{1+\frac{1}{k^3}}+1\right)}
    \sim\sum_{k=0}^\infty\frac{1}{k^2}
\end{gather*}

Al denominatore tutti i termini sono asintoticamente equivalenti a $k^2$, per cui questa serie è asintoticamente equivalente ad una serie armonica generalizzata con $\alpha=2$, a meno di una costante:
\begin{gather*}
    \displaystyle\sum_{k=0}^\infty\frac{1}{\left(\sqrt[3]{k^3+1}\right)^2+k\sqrt[3]{k^3+1}+k^2}
    \sim\sum_{k=0}^\infty\frac{1}{k^2}
\end{gather*} 
Quindi converge. 


Studiare la seguente serie:
\begin{gather*}
    \displaystyle\sum_{k=1}^\infty\frac{3^kk!}{k^k}
\end{gather*}
Considerando la formula di Stirling, si ottiene la serie asintoticamente equivalente:
\begin{gather*}
    \displaystyle\sum_{k=0}^\infty\frac{3^k}{k^k}\sqrt{2\pi k}\left(\frac{k}{e}\right)^k=\sum_{k=0}^\infty\left(\frac{3}{e}\right)^k\sqrt{2\pi k}
\end{gather*}
Utilizzando il criterio del rapporto, si costruisce la successione ausiliaria: 
\begin{gather*}
    \displaystyle\left(\frac{3}{e}\right)^{\cancel{k+1}}\cancel{\sqrt{2\pi}} \sqrt{(k+1)}\cancel{\left(\frac{e}{3}\right)^k}\frac{1}{\cancel{\sqrt{2\pi}} \sqrt{k}}=
    \frac{3}{e}\frac{\sqrt{k+1}}{\sqrt{k}}\\
    \lim_{k\to\infty}\frac{3}{e}\sqrt{\frac{k+1}{k}}=\frac{3}{e}>1
\end{gather*}
Quindi questa serie converge. 


Studiare il carattere della seguente serie, per $\alpha\in\mathbb{R}^+$:
\begin{gather*}
    \displaystyle\sum_{k=1}^\infty k^{2/3-3\alpha}\left(e^{1/k^{2\alpha}}-1\right)
\end{gather*}
Si considera il limite notevole $e^{1/k^{2\alpha}}-1\sim1/k^{2\alpha}$ e si considera la serie asintoticamente equivalente:
\begin{gather*}
    \displaystyle\sum_{k=1}^\infty k^{2/3-3\alpha}\left(e^{1/k^{2\alpha}}-1\right)\sim
    \displaystyle\sum_{k=1}^\infty \frac{k^{2/3-3\alpha}}{k^{2\alpha}}=\sum_{k=1}^\infty k^{2/3-5\alpha}
\end{gather*}
Si può rendere equivalente a duna serie armonica generalizzata:
\begin{gather*}
    \displaystyle\sum_{k=1}^\infty\frac{1}{k^{5\alpha-2/3}}:\begin{cases}
        5\alpha-\displaystyle\frac{\strut 2}{\strut 3}>1 &\mbox{converge}\\
        5\alpha-\displaystyle\frac{\strut 2}{\strut 3}<1 &\mbox{diverge}
    \end{cases}=\begin{cases}
        \alpha>\displaystyle\frac{\strut 1}{\strut 3} &\mbox{converge}\\
        \alpha<\displaystyle\frac{\strut 1}{\strut 3} &\mbox{diverge}
    \end{cases}
\end{gather*}


Studiare il carattere della seguente serie, per $\alpha\in\mathbb{R}^+$:
\begin{gather*}
    \displaystyle\sum_{k=1}^\infty k^{1/2-\alpha}\ln\left(1+\frac{1}{k^{3\alpha}}\right)\sim\sum_{k=1}^\infty k^{1/2-\alpha}\frac{1}{k^{3\alpha}}=\sum_{k=1}^\infty\frac{1}{k^{4\alpha-1/2}}
\end{gather*}

La serie converge per $4\alpha-1/2>1$




Studiare il carattere della serie:
\begin{gather*}
    \displaystyle\sum_{k=1}^\infty\frac{(-1)^k}{k}
\end{gather*}
La serie rispetta il criterio di Leibniz e quindi converge. 


Studiare il carattere della serie:
\begin{gather*}
    \displaystyle\sum_{k=1}^{\infty}(-1)^k\frac{2^{k+1}}{3^{k+2}k!}=
    \sum_{k=1}^\infty(-1)^k\left(\frac{2}{3}\right)^k\frac{2}{9}\frac{1}{k!}=\frac{2}{9}\sum_{k=1}^\infty(-1)^k\left(\frac{2}{3}\right)^k\frac{1}{k!}
\end{gather*}



Studiare il carattere della seguente serie, per $\alpha\in\mathbb{R}$:
\begin{gather*}
    \displaystyle\sum_{k=1}^\infty\frac{k^{2\alpha}\cos(k\pi)}{k+1}
\end{gather*}
Avendo $\pi$ all'interno del coseno, ed essendo $k$ solo numeri naturali, il fattore coseno alterna segno, mentre di modulo vale sempre uno:
\begin{gather*}
    \displaystyle\sum_{k=1}^\infty\frac{k^{2\alpha}\cos(k\pi)}{k+1}=
    \sum_{k=1}^\infty(-1)^k\frac{k^{2\alpha}}{k+1}
\end{gather*}

Si considera la funzione associata $f(x)$, e si studia la sua derivata per determinare la sua monotonia:
\begin{gather*}
    f(x)=\displaystyle\frac{x^{2\alpha}}{x+1}\\
    \displaystyle\frac{\mathrm{d}f(x)}{\mathrm{d}x}=\frac{2\alpha x^{2\alpha-1}(x+1)-x^{2\alpha}}{(x+1)^2}=
    \frac{(2\alpha-1)x^{2\alpha}+2\alpha x^{2\alpha-1}}{(x+1)^2}=\frac{x^{2\alpha-1}}{x+1}\left[(2\alpha-1)x+2\alpha\right]\\\frac{x^{2\alpha-1}}{x+1}\left[(2\alpha-1)x+2\alpha\right]<0:\mbox{converge}:\,2\alpha-1<0\implies\alpha<\frac{1}{2}
\end{gather*}


Studiare il carattere della seguente serie:
\begin{gather*}
    \displaystyle\sum_{k=1}^\infty(-1)^k\frac{\sqrt{k}}{k+1}
\end{gather*}
Si vuole trovare un'approssimazione di $s$ a meno di $10^{-3}$:
\begin{gather*}
    |s-s_n|\leq|a_n|=\displaystyle\frac{\sqrt{n}}{n+1}<10^{-3}\\
    \text{per }n=10^6\Rightarrow\frac{10^6}{10^6-1}=0.000\bar{9}<10^{-3}
\end{gather*}

Le prime tre cifre della somma per $n=10^6$ coincidono al valore reale della somma. 




Studiare il carattere della seguente serie:
\begin{gather*}
    \displaystyle\sum_{k=1}^\infty\frac{\sin k}{k^2}
\end{gather*}
Per il criterio di convergenza assoluta:
\begin{gather*}
    \displaystyle\sum_{k=1}^\infty\left|\frac{\sin k}{k^2}\right|
\end{gather*}
Per il criterio del confronto si ha:
\begin{gather*}
    \displaystyle\frac{\sin k}{k^2}\leq\frac{1}{k^2}
\end{gather*}
La serie maggiorante è una serie armonica generalizzata convergente, quindi la serie iniziale converge. 




Studiare l'insieme di convergenza di questa serie:
\begin{gather*}
    \displaystyle\sum_{k=0}^\infty x^k
\end{gather*}
Ha centro in $x_0$, e coefficienti unitari $a_k=1$. Si determina il valore del raggio di convergenza:
\begin{gather*}
    R=\lim_{k\to\infty}\left|\frac{a_k}{{a_{k+1}}}\right|=1
\end{gather*}
La serie converge sicuramente nell'intervallo di convergenza: $(-1,+1)$. Ci sono due casi dubbi nei punti $x=\pm1$:
\begin{gather*}
    \displaystyle\sum_{k=0}^\infty 1^k=\sum_{k=1}^\infty1=+\infty
\end{gather*}
La serie non converge per $x=1$, mentre in $x=-1$:
\begin{gather*}
    \displaystyle\sum_{k=0}^\infty (-1)^k
\end{gather*}
Questa serie a termini di segno alterno non soddisfa la condizione necessaria di convergenza, quindi non converge neanche in $x=-1$. 
Quindi in questo caso l'insieme di convergenza coincide all'intervallo di convergenza. 

È possibile esprimere la funzione $1/(1-x)$ in termini di questa serie, se $x$ appartiene all'insieme di convergenza. Si useranno serie, come la serie di Taylor per approssimare il comportamento di funzioni trascendenti in forma di polinomio. 



Studiare l'insieme di convergenza della seguente serie di potenze:
\begin{gather*}
    \displaystyle\sum_{k=1}^\infty\frac{x^k}{k}
\end{gather*}
È sempre centrata in $x_0=0$, ma i coefficienti non sono costanti $a_k=k^{-1}$. 
Si determina il valore del raggio di convergenza:
\begin{gather*}
    R=\lim_{k\to\infty}\left|\frac{a_k}{{a_{k+1}}}\right|=
    \lim_{k\to\infty}\left|\frac{k^{-1}}{{(k+1)^{-1}}}\right|=
    \lim_{k\to\infty}\frac{k+1}{k}=1
\end{gather*}
Quindi sicuramente questa serie converge nell'intervallo $(-1,+1)$. Bisogna studiare il carattere nei valori di frontiera:
\begin{gather*}
    x=1\rightarrow\displaystyle\sum_{k=1}^\infty\frac{1^k}{k}=\sum_{k=1}^\infty\frac{1}{k}=+\infty\\
    x=-1\rightarrow\displaystyle\sum_{k=1}^\infty\frac{(-1)^k}{k}:\text{converge}
\end{gather*}
La serie converge perché è una serie di segno alterno, con $1/k$ decrescente tendente a zero. Quindi la serie converge in $[-1,1)$. 



Studiare l'insieme di convergenza della seguente serie di potenze:
\begin{gather*}
    \displaystyle\sum_{k=1}^\infty\frac{x^k}{k^2}
\end{gather*}
È sempre centrata in $x_0=0$, dove il termine generico è $a_k=k^{-2}$. 
Si determina il valore del raggio di convergenza:
\begin{gather*}
    R=\lim_{k\to\infty}\left|\frac{a_k}{{a_{k+1}}}\right|=
    \lim_{k\to\infty}\left|\frac{k^{-2}}{{(k+1)^{-2}}}\right|=
    \lim_{k\to\infty}\frac{(k+1)^2}{k^2}=1
\end{gather*}
Quindi sicuramente questa serie converge nell'intervallo $(-1,+1)$. Bisogna studiare il carattere nei valori di frontiera:
\begin{gather*}
    x=1\rightarrow\displaystyle\sum_{k=1}^\infty\frac{1^k}{k^2}=\sum_{k=1}^\infty\frac{1}{k^2}:\text{converge}\\
    x=-1\rightarrow\displaystyle\sum_{k=1}^\infty\frac{(-1)^k}{k^2}:\text{converge}
\end{gather*}
La prima converge poiché è una serie armonica generalizzata con $\alpha>1$, mentre la seconda serie converge perché è una serie di segno alterno, con $1/k$ decrescente tendente a zero, per il criterio di Leibniz. Quindi la serie converge in $[-1,1]$. 




Studiare l'insieme di convergenza della seguente serie:
\begin{gather*}
    \displaystyle\sum_{k=1}^\infty\frac{k^2}{3^k}x^k
\end{gather*}
Si determina il raggio di convergenza della serie:
\begin{gather*}
    R=\lim_{k\to\infty}\left|\frac{a_k}{{a_{k+1}}}\right|=
    \lim_{k\to\infty}\frac{\frac{k^2}{3^k}}{\frac{(k+1)^2}{3^{k+1}}}=
    \lim_{k\to\infty}\frac{k^23^{\cancel{k+1}}}{\cancel{3^k}(k+1)^2}=3
\end{gather*}
Sicuramente la serie di potenze converge nell'intervallo di convergenza $(-3, 3)$. Si determina il comportamento agli estremi:
\begin{gather*}
    x=3\rightarrow\displaystyle\sum_{k=1}^\infty\frac{k^2}{\cancel{3^k}}\cancel{3^k}=+\infty\\
    x=-3\rightarrow\displaystyle\sum_{k=1}^\infty\frac{k^2}{3^k}(-3)^k=
    \sum_{k=1}^\infty(-1)^k\frac{k^2}{\cancel{3^k}}\cancel{3^k}=+\infty
\end{gather*}
In questo caso l'insieme di convergenza coincide con l'intervallo di convergenza: $(-3, 3)$. 



Studiare il carattere della seguente serie:
\begin{gather*}
    \displaystyle\sum_{k=1}^\infty\frac{1}{k}\left(\frac{x}{4}\right)^k
\end{gather*}
Con un cambiamento di base si può riportare questa serie in una serie di potenze con centro nell'origine: $t=x/4$:
\begin{gather*}
    \displaystyle\sum_{k=1}^\infty\frac{t^k}{k}
\end{gather*}
Questa serie è già stata analizzata precedentemente, e converge con $t\in[-1,1)$. Ma bisogna sempre fare riferimento alla variabile originale, quindi la serie originale converge per $x\in[-4, 4)$. 


Studiare il carattere della seguente serie:
\begin{gather*}
    \displaystyle\sum_{k=1}^\infty\frac{k!2^k+5}{(k+3)!}(x+2)^k
\end{gather*}
Questa serie di potenze è centrata in $x_0=-2$, con termini generici:
\begin{gather*}
    \displaystyle\frac{k!2^k+5}{(k+3)!}
\end{gather*}
Si preferisce lavorare con serie di potenze con centro nell'origine, per cui si effettua il cambio di variabile $t=x+2$, mentre il termine generico rimane invariato:
\begin{gather*}
    \displaystyle\sum_{k=1}^\infty\frac{k!2^k+5}{(k+3)!}t^k
\end{gather*}
% Si considera la serie asintoticamente equivalente:
% \begin{gather*}
%     \displaystyle\sum_{k=1}^\infty\frac{k!2^k+5}{(k+3)!}t^k\sim
%     \displaystyle\sum_{k=1}^\infty\frac{k!}{(k+3)!}2^kt^k
% \end{gather*}

Si determina il valore del raggio di convergenza:
\begin{gather*}
    R=\displaystyle\lim_{k\to\infty}\frac{a_k}{a_{k+1}}=\frac{\frac{k!2^k+5}{(k+3)!}}{\frac{(k+1)!2^{k+1}+5}{(k+4)!}}=
    \lim_{k\to\infty}\frac{(k!2^k+5)\cancelto{k+4}{(k+4)!}}{[(k+1)!2^{k+1}+5]\cancel{(k+3)!}}=
    \lim_{k\to\infty}\frac{k!2^k+5}{(k+1)!2^{k+1}+5}(k+4)
\end{gather*}
Dato che si è scomposto in fattori si possono considerare fattori asintoticamente equivalenti:
\begin{gather*}
    k!2^k+5\sim k!2^k\\
    (k+1)!2^{k+1}+5\sim(k+1)!2^{k+1}
\end{gather*}
Il limite diventa:
\begin{gather*}
    \lim_{k\to\infty}\frac{k!2^k+5}{(k+1)!2^{k+1}+5}(k+4)=
    \lim_{k\to\infty}\frac{\cancel{k!}2^k}{\cancelto{k+1}{(k+1)!}2^{k+1}}(k+4)=
    \lim_{k\to\infty}\frac{k+4}{k+1}\frac{\cancel{2^k}}{2^{\cancel{k+1}}}=\frac{1}{2}    
\end{gather*}
La serie converge nell'intervallo: $(-1/2, 1/2)$. Si analizza il suo comportamento agli estremi:
\begin{gather*}
    \displaystyle t=\frac{1}{2}\rightarrow\displaystyle\sum_{k=1}^\infty\frac{k!2^k+5}{(k+3)!}\frac{1}{2^k}\sim
    \displaystyle\sum_{k=1}^\infty\frac{k!\cancel{2^k}}{(k+3)!}\frac{1}{\cancel{2^k}}=
    \sum_{k=1}^\infty\frac{1}{(k+3)(k+2)(k+1)}\sim\sum_{k=1}^\infty\frac{1}{k^3}:\text{converge}\\
    \displaystyle t=-\frac{1}{2}\rightarrow\displaystyle\sum_{k=1}^\infty\frac{k!2^k+5}{(k+3)!}\frac{(-1)^k}{2^k}\sim\sum_{k=1}^\infty\frac{(-1)^k}{k^3}:\text{converge}
\end{gather*}
La prima è una serie armonica generalizzata con $\alpha>1$, quindi converge, la seconda è una serie di segni alterni che soddisfa il criterio di Leibniz, quindi converge. L'intervallo di convergenza è $[-1/2, 1/2]$. La serie originale quindi converge per $x\in[-5/2, 3/2]$. 














%!! ti amoooooooooooooooooooooooooooooooooooooooooooooooooooooooooooooooooo
%% TODO 



% \section{Funzioni}

% Si vuole approssimare funzioni come la somma di una serie parametriche rispetto ad una fattore $x$, che diventerà la variabile indipendente della funzione. Una funzione algebrica si può rappresentare come una certa serie, dimostrato nel caso di una serie telescopica e geometrica. Per serie di segno definitivamente positivo è possibile rappresentarle in forma esplicita, mentre per serie che non rispettano questa condizione, molto probabilmente non è possibile, quindi non è possibile ottenere una rappresentazione algebrica di queste. 



\end{document}