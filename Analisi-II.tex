\documentclass{article}

\usepackage{cancel}
\usepackage{amsmath,amssymb}
\usepackage[includehead,nomarginpar]{geometry}
\usepackage{graphicx}
\usepackage{amsfonts} 
\usepackage{verbatim}
\usepackage{mathrsfs}  
\usepackage{lmodern}
\usepackage{braket}
\usepackage{bookmark}
\usepackage{fancyhdr}
\usepackage{romanbarpagenumber}
%\usepackage{minted}
%\usepackage{subfig}
\usepackage[italian]{babel}
\usepackage{float}
%\usepackage{wrapfig}
%\usepackage[export]{adjustbox}
\usepackage{contour}
\usepackage[normalem]{ulem}
\allowdisplaybreaks

\setlength{\headheight}{12.0pt}
\addtolength{\topmargin}{-12.0pt}
\graphicspath{ {./Immagini/} }

%% TODO add metadata
\hypersetup{
    colorlinks=true,
    linkcolor=black,
    pdftitle={Appunti di Analisi II},
    pdfauthor={Giacomo Sturm},
    pdfsubject={Analisi II},
    pdfkeywords={}
}

\newsavebox{\tempbox} %{\raisebox{\dimexpr.5\ht\tempbox-.5\height\relax}}


\makeatother
\renewcommand{\contentsname}{Indice}
\numberwithin{equation}{subsection}
\newcommand{\tageq}{\tag{\stepcounter{equation}\theequation}}
\AtBeginDocument{%
    \renewcommand{\figurename}{Fig.}
}
\renewcommand{\ULdepth}{1.8pt}
\contourlength{0.6pt}
\newcommand{\myuline}[1]{%
    \uline{\phantom{#1}}%
    \llap{\contour{white}{#1}}%
}
\fancypagestyle{link}{\fancyhf{}\renewcommand{\headrulewidth}{0pt}\fancyfoot[C]{Sorgente del file \LaTeX disponibile al seguente link: \url{https://github.com/00Darxk/Analisi-II/}}}

\begin{document}

\title{%
    \textbf{Analisi II}  \\ 
    \large Appunti delle Lezioni di Analisi II \\
    \textit{Anno Accademico: 2024/25}}
\author{\textit{Giacomo Sturm}}
\date{\textit{Dipartimento di Ingegneria Industriale, Elettronica e Meccanica \\
Università degli Studi ``Roma Tre"}} 

\maketitle
\thispagestyle{link}

\clearpage


\pagestyle{fancy}
\fancyhead{}\fancyfoot{}
\fancyhead[C]{\textit{Basi di Dati - Università degli Studi ``Roma Tre"}}
\fancyfoot[C]{\thepage}
\pagenumbering{Roman}

\tableofcontents

\clearpage
\pagenumbering{arabic}

\section{Serie Numerica}

Una serie numerica $\{a_n\}$ è una \textit{particolare} successione numerica; 
è una particolare funzione che ha come dominio l'insieme dei numeri naturali $\mathbb{N}$ e come codominio un sottoinsieme $\mathbb{C}$ dell'asse dei numeri reali $\mathbb{R}$. 
\begin{equation}
    \{a_n\}:\mathbb{N}\rightarrow\mathbb{C}\in\mathbb{R}
\end{equation}

Quindi è di interesse studiare il comportamento della serie numerica quando la variabile indipendente $n$ tende all'infinito. Data una sucecssione numerica, si interessa quindi:
\begin{equation}
    \lim_{n\to\infty}a_n
\end{equation}

Si generalizza nel limite di una funzione quando si sostituisce alla variabile $n$ una varibaile indipendente reale $x\in\mathbb{R}$. 

Una successione può convergere ad un numero reale, può divergere positivamente o negativamente, in questi due casi è regolare, quindi ha un limite definito o indefinito. Altrimenti è possibile che il limite non esiste, quindi la sequenza non si stabilisce su nessun numero reale ed oscilla per $n$ tendente ad $\infty$. 

\begin{equation}
    \lim_{n\to\infty}a_n=\begin{cases}
        i\in\mathbb{R}\\
        \pm\infty\\
        \nexists
    \end{cases}
\end{equation}

Si costruisce una serie numerica $\{s_n\}$, partendo da una sequenza nota $\{a_k\}$, si può costruire definendo per ogni valore $n$ il valore di $s_n$ tale sia pari alla somma dei primi $a_n$ elementi della sequenza:
\begin{equation}
    \forall n\in\mathbb{N}:\,
    s_n:=a_1+\cdots+a_n
\end{equation}

$s_n$ viene chiamata somma parziale della serie $\{s_n\}$ e viene definita con il seguente simbolo:
\begin{equation}
    s_n:=\displaystyle\sum_{k=1}^\infty a_k
\end{equation}
La serie numerica è la successione delle somme parziali costruisce sulla successione $\{a_k\}$. 

Una serie numerica si dice convergente se il limite della successione delle somme parziali, costruite a partire dalla successione $a_k$, è pari ad un numero $s$:
\begin{equation}
    \{s_n\}: \mbox{converge se}\,\lim_{n\to\infty}s_n=s\in\mathbb{R}
\end{equation}

Si dice divergente se il limite della successione delle somme parziali tende ad infinito:
\begin{equation}
    \{s_n\}: \mbox{diverge se}\,\lim_{n\to\infty}s_n=\pm\infty
\end{equation}

Se converge o diverge la serie numerica si dice regolare, se invece studiando il comportamento delle somme parziali per $n\to\infty$, il limite della somma non è definito, quindi la serie si dice irregolare:
\begin{equation}
    \{s_n\}: \mbox{irregolare se}\,\lim_{n\to\infty}s_n=\nexists
\end{equation}

Studiare il comportamento della serie vuol dire studiare il comportamento delle somme parziali. La somma parziali si avrà sempre in forma aperta, se fosse possibile esprimere in forma chiusa la forma parziale, allora il limite è di facile calcolo. 

%% ESEMPIO

Considerando il seguente limite:
\begin{equation*}
    \lim_{n\to\infty}(1+2+\cdots+n)
\end{equation*}
Esiste una funzione per esprimere in forma chiusa questa funzione, dimostrabile per induzione:
\begin{equation*}
    \lim_{n\to\infty}\left(\displaystyle\frac{n(n+1)}{2}\right)=+\infty
\end{equation*}

Quindi è facile determinare che si tratta di una serie divergente ad infinito positivo. Questo passaggio non d'ora in avanti non sarà più possibile, basterà stabilire il carattere di una serie, senza sapere il valore a cui converga. Solo nel caso della serie telescopica e della serie geometrica sarà possibile determinare il valore della convergenza. 

\subsection{Serie Telescopica}

Il caso più rappresentativo di una serie telescopica è la seguente:
\begin{equation}
    \displaystyle\sum_{k=1}^\infty\frac{1}{k(k+1)}
\end{equation}

Analizzando la somma parziale $s_n$ fino ad un certo valore $n\in\mathbb{N}$ si ha:
\begin{gather*}
    s_n=\displaystyle\sum_{k=1}^n\frac{1}{k(k+1)}=\frac{1}{2}+\cdots+\frac{1}{n(n+1)}
\end{gather*} 

Utilizzando la tecnica dei fratti semplici si può riscrivere la somma parziale come:
\begin{gather*}
    \displaystyle\sum_{k=1}^\infty\frac{1}{k(k+1)}=\sum_{k=1}^\infty\left(\frac{1}{k}-\frac{1}{k+1}\right)
\end{gather*}

In questo modo calcolando la somma parziale si ottiene:
\begin{gather*}
    s_n=\displaystyle\sum_{k=1}^n\left(\frac{1}{k}-\frac{1}{k+1}\right)=
    \left({1}-\frac{1}{2}\right)+
    \left(\frac{1}{2}-\frac{1}{3}\right)+
    \left(\frac{1}{3}-\frac{1}{4}\right)+
    \cdots+
    \left(\frac{1}{n-1}-\frac{1}{n}\right)+
    \left(\frac{1}{n}-\frac{1}{n+1}\right)
\end{gather*}
Si può notare come l'addendo 1/2 si cancella con l'addendo della differenza successiva, così con 1/3, così per ogni valore della sommatoria, permettendo quindi di scrivere la somma parziale $s_n$ in forma chiusa esprimendo solamente il primo addendo e l'ultimo addendo:
\begin{gather*}
    s_n=\displaystyle\sum_{k=1}^n\left(\frac{1}{k}-\frac{1}{k+1}\right)=
    \left({1}-\frac{1}{2}\right)+
    \left(\frac{1}{2}-\frac{1}{3}\right)+
    \left(\frac{1}{3}-\frac{1}{4}\right)+
    \cdots+
    \left(\frac{1}{n-1}-\frac{1}{n}\right)+
    \left(\frac{1}{n}-\frac{1}{n+1}\right)=
    1-\frac{1}{n+1}
\end{gather*}

Considerando il limite di questa serie si ha:
\begin{equation*}
    \lim_{n\to\infty}\left(1-\displaystyle\frac{1}{n+1}\right)=1
\end{equation*}

\subsection{Serie Geometrica}

La serie geometrica è esprimibile come:
\begin{equation*}
    \displaystyle\sum_{k=0}^\infty x^k
\end{equation*}
Come altre serie dipende da un parametro reale $x\in\mathbb{R}$, questo si chiama ragione della serie. 

Si costruisce la somma parziale della serie:
\begin{equation*}
    s_n=\displaystyle\sum_{k=0}^n x^k=1+x+x^2+\cdots+x^n    
\end{equation*}
Si vuole esprimere questa somma in forma chiusa. Per $x=0$, per ogni valore di $n$, la somma vale sempre $1$, poiché:
\begin{gather*}
    \forall n\in\mathbb{N}: s_n=0^0+0^1+0^2+\cdots+0^n=1
\end{gather*}

Un altro caso analogo con $x=1$, si ha la seguente somma parziale:
\begin{gather*}
    \forall n\in\mathbb{N}: s_n=1+1+1+\cdots+1=1\cdot n
\end{gather*}

Questa serie quindi converge per $x=0$, diverge per $x=1$, mentre per altri valori reali non è ancora noto il suo comportamento. 
Si considera ora il caso generale con una ragione diversa da zero e da uno $\forall x\in\mathbb{R}\setminus\{0,1\}$. 
Si considera noto il valore $s_n$:
\begin{equation*}
    s_n=\displaystyle\sum_{k=0}^n x^k=1+x+x^2+\cdots+x^n    
\end{equation*}
Si moltiplica la somma parziale per la ragione e si sottrae alla somma parziale:
\begin{gather*}
    xs_n=x\cdot\displaystyle\sum_{k=0}^n x^k=\sum_{k=1}^nx^k\\
    (1-x)s_n=1+x+x^2+\cdots+x^n-\left(x+x^2+x^3+\cdots+x^n+x^{n+1}\right)=1-x^{n+1}
\end{gather*}

Della prima somma sopravvive solamente il primo addendo $1$, e della seconda l'ultimo addendo $x^{n+1}$, tutti gli altri si cancellano a vicenda, quindi per ottenere il valore della somma parziale si divide il primo ed il secondo membro per il fattore $1-x$:
\begin{equation}
    s_n=\displaystyle\frac{1-x^{n+1}}{1-x}
\end{equation}
Si provvede ora a calcolare il limite di questa somma parziale, dividendolo per la proprietà di linearità, supponendo che questi due limiti risultatnti non corrispondono ad una forma indeterminata:
\begin{equation*}
    \lim_{n\to\infty}\displaystyle\frac{1-x^{n+1}}{1-x}=\lim_{n\to\infty}\frac{1}{1-x}-\lim_{n\to\infty}\frac{x^{n+1}}{1-x}
\end{equation*}
Il primo fattore non dipende da $n$, per il limite corrisponde a sé stesso, in seguito si mette in evidenza il fattore $1-x$ ottenendo:
\begin{equation*}
    \displaystyle\frac{1}{1-x}\left(1-\lim_{n\to\infty}x^{n+1}\right)
\end{equation*}
Il comportamento di questo limite dipende dalla ragione:
\begin{gather*}
    \lim_{n\to\infty}x^{n+1}=\begin{cases}
        0 &-1<x<1\\
        +\infty  & x>1\\
        \nexists & x\leq-1
    \end{cases}
\end{gather*}
Per $|x|<1$ la funzione è decrescente, e tende a zero, mentre per $x>1$, la funzione è esponenziale e quindi tende ad infinito. Invece per $x<-1$ si considera il limite, mettendo in evidenza il segno negativo:
\begin{gather*}
    \lim_{n\to\infty}(-|x|)^{n+1}=
    \lim_{n\to\infty}\left[(-1)^{n+1}\cdot |x|^{n+1}\right]
\end{gather*}
Il secondo fattore diverge, mentre il primo fattore oscilla di segno in base alla parità di $n$, per cui questa funzione oscilla di ampiezza sempre maggiore, per cui è irregolare. 
Per $x=-1$ il secondo fattore vale uno, mentre il primo fattore si considera comunque, per cui continua ad oscillare alla stessa ampiezza, quindi rappresenza allo stesso modo un comportamento irregolare. 

Noto il comportamento di questo limite allora il valore della serie geometrica in base alla ragione è dato da:
\begin{equation*}
    \displaystyle\sum_{k=0}^\infty=\begin{cases}
        \displaystyle\frac{\strut 1}{\strut 1-x} &-1<x<1\\
        +\infty & x\geq1\\
        \nexists & x\leq-1
    \end{cases}
\end{equation*}

Questa serie ha un addendo in più poiché parte da $k=0$, per cui esprimendola come una serie partendo da $k=1$:
\begin{equation*}
    \displaystyle\sum_{k=1}^\infty=\displaystyle\sum_{k=0}^\infty-1
\end{equation*}
Il valore di questa successione quindi si comporta analogamente al precedente, considerando l'addendo in più:

\begin{equation*}
    \displaystyle\sum_{k=1}^\infty=\begin{cases}
        \displaystyle\frac{\strut 1}{\strut 1-x}-1 &-1<x<1\\
        +\infty & x\geq1\\
        \nexists & x\leq-1
    \end{cases}
\end{equation*}


\end{document}